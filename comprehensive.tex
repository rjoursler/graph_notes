\documentclass[12pt]{article}
\usepackage{amsmath}
\usepackage{amsthm}
\usepackage{amssymb}
\usepackage[top=1in,bottom=1.25in,left=1.25in, right=1.25in]{geometry}
\newtheorem{theorem}{Theorem}
\newtheorem{definition}{Definition}
\newtheorem{problem}{Problem}

\begin{document}
\section{Introduction} This document contains theorems and sample
problems for the Graph Theory Comprehensive exam at Arizona State
University. Most of the theorems are found in Diestel's Graph Theory
and when a theorem number is mentioned, that number refers to the 3rd
edition. Most of the proofs that are provided are generally similar to
proofs in Diestel's Graph Theory, or West's Introduction to Graph
Theory. This document was originally formed while studying for the ASU
graph theory comprehensive and as such is incomplete. Some proofs are
missing, some proofs are better described as proof summaries, and the
occasional proof may be incorrect. Any corrections/additions to this
document are welcome and can be submitted to this documents GitHub
page located at https://github.com/rjoursler/graph\_notes.

\section{Basic Theorems}
\begin{theorem} (1.4.2) If $G$ is non-trivial, then
  $\kappa(G) \leq \lambda(G) \leq \delta(G)$.
\end{theorem}
\begin{proof} Let $G$ be a non-trivial graph. To show that
  $\kappa(G) \leq \lambda(G)$, let $C$ be a a set of edges with
  $|C| = \lambda(G)$ which separates $G$ into sets $A$ and $B$. Note
  $|A|, |B| \geq \lambda(G)$. By deleting the ends of edges in $C$
  from $G$ in such a way that $A$ and $B$ are still non-empty we get
  $\kappa(G) \leq \lambda(G)$. If this cannot be done, then
  $\lambda(G) = 1$ and $|A| = |B| = 1$, implying $G = K_2$ in which
  the claim still holds. For the fact that
  $\lambda(G) \leq \delta(G)$, note than any vertex $v$ can be
  disconnected from the graph by deleting all edges which contain $v$.
\end{proof}

\begin{theorem} (1.5.6) Every connected graph contains a normal spanning tree, with any
  specified vertex as its root.
\end{theorem}
\begin{proof} The depth first search algorithm will yield a normal
  tree starting from any vertex. Since $G$ is connected, every vertex
  will be visited, thus the spanning tree is normal.
\end{proof}

\begin{theorem} (1.6.1) A graph is bipartite iff it contains no odd cycle.
\end{theorem}
\begin{proof} The forward direction follows immediately since an odd
  cycle is $3$-partite.\\
  For the backward direction, assume $G$ is a graph which contains no
  odd cycle. WLOG we may assume that $G$ is connected, otherwise we
  can reduce the problem to independently partitioning all the
  connected subgraphs of $G$. Let $v \in E(G)$. Define the set $A$ to
  be the set of vertices $a$ such that there is an even path from $v$
  to $a$ and $B$ to be vertices $b$ such that there is an odd path
  from $v$ to $b$. Since $G$ is connected $A \cup B = V(G)$. Also, if
  the edge $xy$ exits within $A$ or $B$, then $x,y \in A$ and
  $x,y \in B$. Finally, note that $A \cap B = \emptyset$ otherwise an
  odd cycles could be found. Thus $G$ is bipartite with bipartition
  $A,B$.
\end{proof}

\begin{theorem} (1.4.3 Maders Theorem) Let $0 \neq k \in
  \mathbb{N}$. Every graph $G$ with $d(G) \geq 4k$ has a $(k + 1)$
  connected subgraph $H$ such that $\epsilon(H) \geq \epsilon(G) - k$.
\end{theorem}
\begin{proof} Consider all graphs $G' \subseteq G$ such that
  $||G'|| \geq \epsilon(G) (|G'| - k)$. Such graphs exists since $G$
  satisfies these conditions. Let $H$ be such a graph such that $|H|$
  is minimum. By the minimality of $H$, $\delta(H) \geq \epsilon(G)$,
  otherwise we could delete such vertex to find a smaller graph
  $H'$. Thus $|H| > \epsilon(G)$. Using this fact gives
  \[
    \epsilon(H) = \dfrac{||H||}{|H|} > \epsilon(G)(1 - \dfrac{k}{|H|})
    \geq \epsilon(G) - k
  \]
  as desired.

  All that is left to show is that $H$ is $k + 1$-connected. Assume to
  the contrary. Then $H$ has a separating set $X$ of size at most $k$
  which separates sets $A$ and $B$. Consider the graph $H[A \cup
  X]$. Because of the minimum degree in $H$, for any vertex $a \in A$,
  the degree of $a$ in $H[A \cup X]$ is at least $\delta(H)$ implying
  $|H[A \cup X]| \geq 2k$. The same applies for $H[B \cup X]$. Thus we
  get that
  \[
    ||H[A \cup X] || \leq \epsilon(G)(|H[A \cup X] - k)
  \]
  and
  \[
    ||H[B \cup X] || \leq \epsilon(G)(|H[B \cup X] - k)
  \]
  This implies that
  \[
    \begin{array}{rcl} ||H|| & \leq & ||H[A \cup X]|| + ||H[B \cup X]||\\
                             & \leq & \epsilon(G) (|H[A \cup X] + |H[B
                                      \cup X]| - 2k)\\
                             & \leq & \epsilon(G) (|H| - 2k + |X|)\\
                             & \leq & \epsilon(G) (|H| - k)\\
    \end{array}
  \]
  a contradiction.
\end{proof}

\begin{theorem} (1.8.1) A connected graph is Eulerian iff every vertex
  has even degree.
\end{theorem}
\begin{proof} To begin with, assume $G$ is an Eulerian graph with
  Eulerian tour $W$. Note that every vertex must be entered as many
  times as it is exited in the tour $W$ thus every vertex in $G$ must
  have even degree.\\
  For the reverse direction, let $G$ be a connected graph such that
  every vertex has even degree. Let $W = v_0 v_1 \ldots v_\ell$ be a
  maximum length walk such that each edge appears at most once. Since
  the degree of $v_0$, $v_0 = v_\ell$. Thus $W$ is a closed walk. If
  $G$ has an edge outside of the walk then we can extend the walk, a
  contradiction.
\end{proof}

\begin{theorem} (Dirac's Theorem 10.1.1) Every graph with $n \geq 3$
  vertices and a minimum degree at least $n/2$ has a Hamilton cycle.
\end{theorem}
\begin{proof} To begin with, note that $G$ must be connected, because
  the maximum degree in the smallest component otherwise would be less
  than $n/2$.  Let $P = v_0, \ldots, v_k$ be a maximum path in $G$. By
  the maximality, of $P$, $N(v_0), N(V_k) \in V(P)$. If there is an
  edge from $v_0$ to $v_k$ then we get a cycle $C$ on the vertices of
  $P$. Otherwise, since the size of $P$ is smaller than $n$, by pigeon
  hole, there must exists vertices $x_i$ and $x_{i + 1}$ such that
  $v_0 x_{i + 1}$ and $v_k x_i$ are edges creating a different cycle
  $C$ on the vertices of $P$. If $C$ is not spanning, then by the
  connectivity of $G$, $C$ is connected to a vertex not in $C$, which
  forms a longer path contradicting maximality of $P$.
\end{proof}

\begin{theorem} (Konig's Theorem 2.1.1) Let $G$ be a bipartite
  graph. Then the size of the maximum matching in $G$ is equal to the
  size of a minimum vertex covering in $G$
\end{theorem}
\begin{proof} Let $G$ be a bipartite graph with bipartition
  $(A,B)$. To begin with note that the size of the minimum covering
  has to be at least as large as the maximum matching in order to
  cover every edge in a maximum matching. Now for equality, consider a
  maximum matching $M$. We will create a covering $C$ by choosing a
  one vertex from each edge of the matching, picking a vertex from $B$
  if an alternating path with an unmatched vertex in $A$ ends on that
  vertex, and picking the vertex from $A$ otherwise. Now we will show
  that the set of vertices $C$ forms a covering.

  Let $ab \in E$ be any edge in $G$. We will show that either $a$ or
  $b$ lies in $C$. If $ab \in M$, then this holds trivially, so we may
  assume that $ab \notin M$. To begin with, assume that $a$ is not
  matched by $M$. Then there must exists an edge $a'b \in M$, and
  since $b$ is the end of the alternating path $ab$, $ab$ is covered
  by $C$. Otherwise there exists an edge $ab' \in M$. If $a \in C$,
  then we are done. Otherwise $b' \in C$ by construction, so there is
  an alternating path ending in $b'$. But this creates an alternating
  path ending in $b$ as well. Since $M$ is maximum, that implies $b$
  is matched with an edge $a'b \in M$ and thus $b \in C$. Therefore
  $C$ is an edge cover proving the theorem.
\end{proof}

\begin{theorem} (Hall's Theorem 2.1.2) A bipartite graph $G$ with
  bipartition $(A,B)$, contains a matching of $A$ iff
  $|N(S)| \geq |S|$ for all $S \subseteq A$.
\end{theorem}
\begin{proof} To begin with, note that a matching of $A$ immediately
  that $|N(S)| \geq |S|$ for all $S \subseteq A$. Now to show the
  other direction, assume $G$ is a bipartite graph with bipartition
  $(A,B)$ such that $N(S) \geq |S| $ for all $S \subseteq A$. Let $M$
  be a maximum matching of $A$ and assume that $A$ is not
  matched. Therefore there exists a vertex $a \in A$ not matched by
  $M$. Let $A'$ be the set of vertices reachable from $a$ by an
  alternating path, and let $B'$ be the penultimate vertices in all
  such paths. Since very vertex of $A'$ corresponds to at least one
  vertex of $B'$ we get that $|A'| \leq |B'|$. Similarly, every
  neighbor of a vertex $a' \in A$, must appear in $B'$, otherwise we
  could find either a new alternating path or an augmenting path. Thus
  $|A'| = |B'|$ by the assumption.  Thus by the assumption, if
  $S = A' \cup \{ a\}$, there must be an edge $ab$ such that
  $b \notin B'$. Note that $b$ must be matched otherwise we could add
  $ab$ to the matching $M$ contradicting maximality. But then $b$ is
  the penultimate vertex some some alternating path starting at $a$
  and that $b \in B'$, a contradiction. Therefore $M$ must match all
  of $A$.
\end{proof}

\begin{theorem} (2.1.3) If $G$ is a $k$-regular bipartite graph with
  $k \geq 1$, then $G$ has a 1-factor
\end{theorem}
\begin{proof} Since $G$ is $k$-regular, then $|A| = |B|$. Thus if we
  can find a perfect matching of $A$ onto $B$, then we are done. Let
  $S \subseteq A$. Note that the degree sum of vertices in $N(S)$ is
  at least as large as the degree sum of the vertices of $S$. Since
  $G$ is $k$ regular we get that $k |S| \leq k N(S)$. Thus by Hall's
  Marriage theorem we get a matching of $A$ onto $B$.
\end{proof}

\begin{theorem} (Petersen's Theorem 2.1.5) Every regular graph of
  positive even degree has a 2-factor.
\end{theorem}
\begin{proof} Let $G$ be a regular graph of positive even
  degree. Since every vertex has even degree, there exists an Eulerian
  tour $W$. Use this tour to induce an orientation of $G$ and then
  create to copies of $V(G)$, $V^-$ and $V^+$. Let $v_- v_+$ be an
  edge between those two sets iff $v_- v_+$ is an edge in the
  orientation of $G$. Note that by definition this new graph of $V^-$
  and $V^+$ is $k$ regular, thus there exists a matching of $V^-$ onto
  $V^+$. Projecting this matching back into $G$ then provides a
  2-factor of $G$.
\end{proof}

\begin{theorem} (Stable Matching Theorem 2.1.4) Let $G$ be a bipartite
  graph. For every set of preferences, $G$ has a stable matching.
\end{theorem}
\begin{proof} Construct a matching greedily as follows. Given a
  matching $M$, for a given vertex, $b$ say that a vertex $a \in A$ is
  acceptable if $a$ is unmatched or of $b$ prefers $a$ over its
  current matched neighbor. Starting with the empty matching, pick a
  vertex $a$ such that $a$ is unmatched but acceptable to some
  $b$. Then add the edge $ab$ to the matching for the $b$ that $a$
  most prefers, while deleting any edge $bc$ already existing in our
  matching. Since this always increases the happiness of a vertex $b$,
  this algorithm must end. Note at each step, no vertex $a$ desires to
  pick a different vertex $b$ since it chose the best possible one
  when the edge $ab$ was added and once a vertex is matched in $b$, it
  will always be matched. Thus the matching is stable.
\end{proof}

\begin{theorem} (Tuttes's Theorem 2.2.3) Every graph $G$ contains a
  subset $S$ such that $S$ is matchable to $C_{G - S}$ and every
  component in $G-S$ is factor critical.
\end{theorem}
\begin{proof} The proof is by induction. For the base case, not that
  when $|G| = 0$, the claim holds. For the induction, pick $G$ and
  consider all sets $T$ such that $q(G-T) - |T|$ is maximized. Let $S$
  be the the the set for which $|T|$ is also maximized. Now note that
  there must be no even sized components in $G - S$, otherwise we
  could get a larger set $S'$ by moving a vertex from the even
  component into $S$. This new set $S'$ still maximizes
  $q(G-S) - |T|$, thus contradicting the maximality of $S$.

  Now assume that there exists a component $C$ in $G - S$ such that
  $|C|$ is not factor critical. Let $C' = C-c$ where $c$ is a vertex
  such that $C'$ has no $1$-factor. Then by the inductive hypotheses
  there exists a set $S' \subseteq C$ such that $q(C' - S') >
  |S'|$. Since $|C'|$ is even, the parity of $q(C'-S')$ and $|S'|$ are
  the same, thus $q(C'-S') \geq |S'| + 2$. But then for
  $T = S \cup \{c\} \cup S'$,
  $q(G-T) - |T| \geq q(G-S) - |S| - 1 - 1 + q(G - S') - |S'| \geq
  q(G-S) - |S|$, once again contradicting choice of $S$.

  Finally, all that is left is to show that $S$ is matchable onto the
  components of $G - S$. Assume to the contrary, then by Halls theorem
  there exists a set $S' \subseteq S$ such the set $S'$ is connected
  to fewer components than $|S'|$. But than deleting $S'$ from $S$
  destroys fewer odd components from $G-S$ than it removes from $S$,
  thus increasing the size of $q(G-T) - |T|$, contradicting the choice
  of $S$ again.
\end{proof}

\[
  s_k := 4k(\log k + \log \log k + 4) if k \geq 2 1 if k \leq 1
\]

\begin{theorem} (2.3.1) Let $k \in \mathbb{N}$, and let $H$ be a cubic
  multigraph. If $|H| \geq s_k$, then $H$ contains $k$ disjoint
  cycles.
\end{theorem}
\begin{proof} The proof will be by induction. Note the claim is
  trivial if $k \leq 1$. Let $C$ be a shortest cycle in $H$. If $C$ is
  a loop $xx$, then the only vertex meeting $C$ is $x$. Since
  $d(x) = 3$, the graph $H - C$ contains $|C|$ vertices with degree
  exactly one. Thus none of these vertices are in a cycle in $H -
  C$. Thus taking $H - V(C)$ only contains vertices of degree $2$ or
  degree $3$. Suppressing the vertices of degree $2$ results in
  another cubic multigraph. In this process, at most $2|C|$ vertices
  were deleted from $H$ and at least $|C|$ vertices were
  deleted. Since $|C| \leq 2 \log |H|$, we get that
  $|H'| \geq s_k - 4 \log s_k = s_{k -1 }$ for $k \geq 2$.
\end{proof}

\begin{theorem} (Erdos, Posa 2.3.2) There is a function
  $f: \mathbb{N} \to \mathbb{R}$ such that, given any
  $k \in \mathbb{N}$, every graph contains either $k$ disjoint cycles
  or a set of at most $f(k)$ vertices meeting all of its cycles.
\end{theorem}
\begin{proof} Let $G$ be any graph with fewer than $k$ disjoint
  cycles. We may assume it has a cycle. Let $H$ be the maximal
  subgraph of $G$ for which ever vertex has degree either $2$ or
  $3$. Let $U$ be the vertices of $H$ with degree at least $3$. Let
  $C$ be the set of all cycles that avoid $U$ and meet $H$. Since $H$
  is maximal, each cycle must meet at exactly one point. Thus the set
  $Z$ disconnects all such cycles. Let $Y$ be a set of vertices with
  exactly one vertex from the two regular components of $H$ that avoid
  $Z$. Note each vertex in $Z$ and $Y$ corresponds to a disjoint
  cycle, thus $|Y \cup Z| \leq k - 1$. Now consider any cycle $C$
  which avoids $X$. Then $C$ is not a component, so $C$ must meet
  $U$. From $2.3.1$, if $|U| \geq s_k$, $H$ contains $k$ disjoint
  cycles. Otherwise $U \cup Y \cup X$ meets every cycle in $G$ and
  this set is smaller than $s_k + k - 1$.
\end{proof}

\begin{theorem}{2.4.3} Let $G$ be a graph and $F = (F_1, \cdots F_k)$
  be a $k$-tuple of edge-disjoint forests such that the number of
  edges contained in the forests of $F$ is maximum. Then for every
  edge $e \notin F'_i$ for all $i$, for some $F'$ obtained by a series
  of edge replacements, there exists a set $U$ with $e \subseteq U$
  such that $U$ is connected in all $F_i$.
\end{theorem}
\begin{proof} Let $E^0$ be the set of all edges which can be removed
  from all $F_i$ by series of edge replacement and define
  $G^0 = (V, E^0)$. Let $C^0$ be a component in $G^0$. We claim that
  $U = V(C^0)$ is connected in all $F_i$.

  Consider a family of trees $F'=(F'_1, \ldots, F'_k)$ obtained by
  replacing an edge of $F_i$. We claim that if $x,y$ are the end of a
  path in $F'_i \cap C^0$, then $x F_i y \subseteq C^0$. To see why,
  let $e = vw$ be the new edge in $F'$. We may assume that
  $e \in x F'_i y$, because otherwise the claim is obviously. Thus it
  suffices to show that $v F_i w \subseteq C^0$. Let $e'$ be any edge
  in $v F_i w$. Since we could replace $e'$ by $e$ we have that
  $e' \in E^0$. Since this applies to all $e'$ we get that
  $vF_i w \subseteq C^0$.

  To finish the proof of this theorem, we can obtain a new forest
  tuple $F'_i$ by replacing $F_i$ using $e$. Since $e$ can be viewed
  as an $xF'_i y$ path in $C_0$, we get that $x F_i y \subseteq
  C_0$. As this applies for all $i$ and edges in $C_0$, $U$ must be
  connected in every $F_i$.
\end{proof}

\begin{theorem} (Nash-Williams 2.4.1) A multigraph contains $k$
  edge-disjoint spanning trees iff for every partition $P$ of its
  vertex set it has at least $k(|P|-1)$ cross-edges.
\end{theorem}
\begin{proof} For the forward direction, note that the graph obtained
  by contracting the parts in a partition $P$ into single vertices
  results in a connected graph. Thus each edge-disjoint spanning tree
  has at least $|P| - 1$ vertices crossing between parts, resulting in
  at least $k(|P| - 1)$ crossing edges.

  For the reverse direction will be proved by induction on $|G|$. Note
  for $|G| = 2$, the assertion holds. Now pick a $k$-tuple of edge
  disjoint forests $F = (F_{1},\cdots F_{k})$ such that the number of
  edges is maximized. If each $F_{i}$ is a tree, then we are
  done. Otherwise we have
  \[
    ||F|| = \sum_{i = 1}^k ||F_{i}|| < k(|G| - 1)
  \]
  On the other hand we know that $||G|| \geq k (|G| - 1)$ by the
  assumption applied when $G$ is partitioned into $|G|$ parts. Thus
  there exists an edge $e \notin F^0$. Thus by the previous lemma,
  there exists a set $U$ with $|U| \geq 2$ that is connected in all
  $F_i$. By the inductive hypothesis, $G\backslash U$ contains
  $k$-edge disjoint trees $T_1 \cdots T_k$. Replacing the vertex $v_U$
  in each $T_i$ with the corresponding graph $F_i[U]$ in $G$ then
  gives $k$ edge-disjoint spanning trees in $G$.
\end{proof}

\begin{theorem} (Nash-Williams 2.4.4) A multigraph $G = (V, E)$ can be
  partitioned into at most $k$ forests iff $||G[U]|| \leq k(|U| - 1)$
  for every non-empty set $U \subseteq V$.
\end{theorem}
\begin{proof} For the forward direction, note that on any set $U$,
  there are at most $|U| - 1$ edges associated with any forest. As
  such, $||G[U]|| \leq k (|U| - 1)$ for all sets $U$.

  The other direction will be proved by contrapositive. Assume that
  $G$ cannot be partitioned into at most $k$ forests. Then for any
  $k$-tuple $F = (F_1, \cdots F_k)$ with the number of edges
  maximized, $F$ does not form a partition. But then there exists an
  edge $e$ not in any $F_i$, so by a previous lemma there exists a set
  $U$ such that $|U| \geq 2$ and such that $U$ is connected in every
  $F_i$. But then $||G[U]|| \geq k(|U| - 1) + 1$.
\end{proof}

\begin{theorem} (Gallai \& Milgram 2.5.1) Every directed graph $G$ has
  a path cover $\mathcal{P}$ and independent set
  $\{ v_P | P \in \mathcal{P} \}$ of vertices such that $v_P \in P$
  for every $P \in \mathcal{P}$.
\end{theorem}
\begin{proof} Let
  $ter(\mathcal{P}) = \{ v_0, \cdots v_{|\mathcal{P}|} \}$ denote the
  set of vertices ending the directed paths in $\mathcal{P}$. We will
  prove by induction on $|G|$ that if $|ter(\mathcal{P})|$ is minimal
  then there exists such an independent set as is claimed. Clearly
  this is satisfied when $|G| = 0$. Let $\mathcal{P}$ be a path cover
  over $G$ with $|ter(\mathcal{P})|$ minimal. If $ter(\mathcal{P})$ is
  independent, then we are done. Otherwise we may assume that
  $v_1 v_0$ is an edges. Let $v$ be the vertex preceding $v_0$ in the
  path ending in $v_0$. Define $G' = G - v_0$. Now consider another
  path family $\mathcal{P}'$ covering $G'$ for which
  $ter(\mathcal{P'}) = \{ v, v_1, \ldots, v_{|\mathcal{P}|} \}$ (which
  obviously exists). Now if this path family in $G'$ contains an
  independent set then that set is obviously independent in $G$ as
  well so we are done. Thus it suffices to show that
  $|ter(\mathcal{P}')|$ is minimal.

  For contradiction, assume there exists a family $\mathcal{P}''$ such
  that $ter(\mathcal{P}'') \subset ter(\mathcal{P}')$. If $v$ or $v_1$
  is in $ter(\mathcal{P}'')$, then this path cover can be extended to
  a path cover in $G$ which contradicts the minimality of
  $ter(\mathcal{P})$. Hence
  $ter(\mathcal{P}'') \subseteq \{ v_3, \ldots, v_{|\mathcal{P}|}
  \}$. But then $\mathcal{P}''$ and $v_1$ form a path cover of $G$
  contradicting the minimality of $|ter(\mathcal{P})|$ again.
\end{proof}

\begin{theorem} (3.1.3) A graph is 2-connected iff it can be
  constructed from cycles by successively adding $H$-paths to graphs
  $H$ already connected.
\end{theorem}
\begin{proof} Note that any graph constructed as above must clearly be
  $2$-connected. For the converse, let $G$ be a $2$-connected
  graph. Since $G$ is $2$-connected it contains a cycle and thus $G$
  has a maximum sized subgraph $H$ such that $H$ can be constructed by
  successively adding $H$-paths to a cycle. Note that $H$ must be an
  induced graph, because any edge not in $H$, but between vertices of
  $H$ is an $H$-path. Finally, form connectivity there must exists an
  edge $vw$ such that $v \in V(H)$ and $w \notin V(H)$. Since $G$ is
  $2$-connected there must be a path $P$ from $w$ to $H$ in $G-v$, but
  then $vwP$ is an $H$ path, contradicting the maximality of $H$.
\end{proof}

\begin{theorem} (3.2.1) If $G$ is $3$-connected and $|G| > 4$, then
  $G$ has an edge $e$ such that $G/e$ is again $3$-connected.
\end{theorem}
\begin{proof} Assume there exists no edge $e$ such that $G/e$ is $3$
  connected. Then for any edge $xy$, there must exists a vertex $z$
  such that $\{v_{xy}, z\}$ is a separator in $G/e$, and therefore
  ${x,y,z}$ must be a separator in $G$ since $G$ is
  $3$-connected. Pick an edge $xy$ such that the separator $\{x,y,z\}$
  produces a component $C$ of minimum size. Since $\{x,y,z\}$ is a
  minimum separator, there must exists a vertex
  $v \in N(z) \cap V(C)$, because otherwise $\{x,y\}$ would be a
  separator for $C$. Let $w$ be the vertex associated with the edge
  $zv$ so that $\{z, v, w\}$ is a minimum separator in $G$. Since $xy$
  is an edge, there must exists a component $D$ of $G - \{z,v,w\}$
  such that $V(D) \cap \{x,y\} = \emptyset$, since $x$ and $y$ are not
  separated by $\{z, v, w\}$. Also, since
  $N(v) \cap V(D) \subseteq V(C)$, we get that $D$ is contained in
  $C$, contradicting the minimality of $C$.
\end{proof}

\begin{theorem} (3.2.2) A graph $G$ is $3$-connected iff there exists
  a sequence $G_0, \ldots, G_n$ of graph with the following
  properties:
  \begin{itemize}
  \item[(i)] $G_0 = K^4$ and $G_n = G$
  \item[(ii)] There exists an edge $xy \in G_{i+1}$ with
    $d(x), d(y) \geq e$ and $G_i = G_{i + 1} /xy$ for every $i < n$.
  \end{itemize}
\end{theorem}
\begin{proof} Proof follows immediately from $3.2.1$.
\end{proof}

\begin{theorem} (3.2.3) The cycle space of a $3$-connected graph is
  generated by its non-separating induced cycles.
\end{theorem}
\begin{proof} Let $G$ be a $3$-connected graph with $|G| = n$. We will
  show that each cycle $C$ is a sum of non-separating induced cycles
  by induction on $k(C) = n - b$ where $b$ is the order of the largest
  component in $G-C$, and $b = 0$ if $V(G) = V(C)$. Since there are no
  cycles with $k(C) = 0$, the base case trivially holds. Now let $C$
  be a cycle assume the inductive hypothesis holds for all cycles $C'$
  with $k(C') < k(C)$.

  To begin with, if $C$ is a spanning cycles, then it contains a chord
  $e$ because $G$ is $3$-connected. But then $C$ is the sum of two
  non-spanning cycles $C_1,C_2 \subseteq C + e$ with $e \in C_1,
  C_2$. Since $k(C_1),k(C_2) < n = k(C)$ due to the largest component
  of $G -C_i$ having order at most $n - |C_i|$, we are done by the
  inductive hypothesis.

  Assume that $G-C$ is non-empty and let $B$ be the largest component
  of $G-C$. Suppose there a $C$-path $P$ from $u$ to $v$ in $G-B$ such
  that both $uCv$ paths contain an internal vertex in $N(B)$. Then
  there are two cycles $C_1,C_2 \subset C \cup P$ with
  $P \subset C_1, C_2$ such that $C$ is the sum of $C_1$ and
  $C_2$. Also, there is a component in $G - C_i$ properly containing
  $B$, so we are done by induction.

  Assume there is no such path $P$. Then all vertices $v$ have a
  neighbor in $B$. Assume to that contrary and let $v$ be a vertex
  with $v \notin N(B)$. Let $x,y$ be $v$'s two closest neighbors with
  $x,y \in N(B)$ and let $Q$ be the path from $x$ to $y$ on $C$
  passing through $v$. Since $G$ is $3$-connected, $C-Q$ is nonempty
  and there must be a path $P$ from $Q-x-y$ to $C-Q$ in $G - x -
  y$. But this path has the property described above, a
  contradiction. Thus all vertices in $C$ contain a neighbor in
  $B$. Finally note that $C$ cannot contain a chord since that chord
  forms one of the banned paths.

  Thus $C$ is an induced cycles. If $C$ is non-separating then we are
  trivially done. Otherwise there exits another component $B' \neq B$
  in $G-C$. Let $P = u, \cdots v$ be a $C$-path through $B'$ and let
  $Q$ be a $C-P$ path. Then there exists cycles
  $C_1, C_2, C_3 \subset C \cup P \cup Q$ which sum to $C$. As before
  $G - C_i$ has a component properly containing $B$, so we are done by
  the inductive hypothesis.
\end{proof}

\begin{theorem} (3.3.1 Menger's Theorem) Let $G=(V,E)$ be a graph and
  $A,B \subseteq V$. Then the minimum number of vertices separating
  $A$ form $B$ in $G$ is equal to the maximum number of disjoint
  $A$-$B$ paths in $G$.
\end{theorem}
\begin{proof} The proof is by induction. If $||G|| = 0$, then the
  number of $A$-$B$ paths must equal $|A \cap B|$ since there are
  $|A \cap B|$ trivial $A$, $B$ paths. Otherwise we have an edge
  $e = xy$. Let $k$ be the size of the minimum separator separating
  $A$ and $B$. Consider the graph $G / e$. If $v_c$ is the vertex
  created by contraction, let $v_c \in A$ if $x$ or $y$ was in $A$
  (and similarly for $B$). Then $G/e$ contains a minimum separator
  $Y$. If $|Y| \geq k$, then we are done by induction. Thus
  $|Y| \geq k$. Note that $v_c \in Y$, otherwise $Y$ would be a
  separator of $G$, a contradiction. Define $X = Y - v_c + x +
  y$. Then $X$ is an $A,B$ separator in $G$ on exactly $k$
  vertices. Now consider $G-e$. By the inductive hypothesis there must
  be $k$ disjoint $A-X$ paths and $k$ disjoint $B-X$ paths. These sets
  of paths must be disjoint as well since $X$ is a
  separator. Combining this paths produces $k$ disjoint $A$-$B$ paths.
\end{proof}

\begin{theorem} (3.3.6 Menger's Theorem (global version))
  \begin{itemize}
  \item A graph is $k$-connected iff it contains $k$-independent paths
    between any two vertices.
  \item A graph is $k$-edge connected iff it contains $k$ edge
    disjoint paths between any two vertices.
  \end{itemize}
\end{theorem}
\begin{proof} Follows immediately from 3.3.1.
\end{proof}

\begin{theorem} (3.4.1 Maders' Theorem) Given a graph $G$ with an
  induced subgraph $H$, there are always $M_G(H)$ independent
  $H$-paths in $G$.
\end{theorem}
\begin{proof} No proof given in Diestel
\end{proof}

\begin{theorem} (3.4.2) Given a graph $G$ with an induced subgraph
  $H$, there are at least $1/2 \kappa_G (H)$ independent $H$-paths and
  at least $1/2 \lambda_G(G)$ edge disjoint $H$-paths in $G$.
\end{theorem}
\begin{proof}
\end{proof}

\begin{theorem} (3.5.1) There exists a function
  $h: \mathbb{N} \to \mathbb{N}$ such that every graph of average
  degree at least $h(r)$ contains $K^r$ as a topological minor for
  every $r \in \mathbb{N}$.
\end{theorem}
\begin{proof} The proof will be by induction. To begin with, note that
  $h(r) = 1$ holds for $r \leq 2$. We will show by induction on
  $m = r \cdots {r \choose 2}$ that if $G$ has average degree $2 ^ m$,
  then $G$ has a topological copy of $X$ where $X$ has $r$ vertices
  and $m$ edges. When $m = {r \choose 2}$ this will immediately imply
  the claim. If $m = r$, then $G$ contains a cycle of length at least
  $\epsilon(G) + 1 \geq 2^{r -1} + 1 \geq r + 1$. Letting $X = C^r$
  suffices. Now assume $m > r$ and the claim holds for any smaller
  value. If $G$ is not connected, then there exists a component of $G$
  with at least the same average degree, and we are done by
  induction. Thus $G$ is connected. Let $U$ be a maximum set of
  vertices such that $U$ is connected and
  $\epsilon(G \backslash U) \geq 2 ^ {m - 1}$. Note such a set exists
  because deleting a vertex cannot decrease the average degree of $G$
  to below $2 ^ {m - 1}$. Also, since $G$ is connected, $N(U)$ is
  non-empty. Let $H = G[N(U)]$. Not $\delta(H) \geq 2 ^ {m -1}$,
  otherwise we could add another vertex to $U$, contradicting $U$'s
  maximality. Thus by the induction hypothesis, $H$ contains a $TY$,
  with $|Y| = r$, and $||Y|| = m - 1$. Let $x,y$ be two non-adjacent
  branch vertices. Since $G$ is connected there is a $x-y$ path in $U$
  creating a $TX$ on $m$ vertices.
\end{proof}

\begin{theorem} (3.5.2) There is a function
  $f: \mathbb{N} \to \mathbb{N}$ such that every $f(k)$-connected
  graph is $k$-linked, for all $k \in \mathbb{N}$.
\end{theorem}
\begin{proof} Let $G$ be $h(3k) + 2k$ connected, where $h$ is as
  defined in the previous Lemma. Let $X$ be the set of vertices
  $s_1 \cdots s_k$ and $t_1 \cdots t_k$ we would like to connect. Then
  $G - X$ contains $K$ a copy of $TK^{3k}$ since
  $\kappa(G) \geq h(3k) \geq \delta(G)$. Let $U$ be the branch
  vertices of a $K$. Since $G$ is $2k$ connected there are $2k$
  independent paths from $s_1 \cdots s_k$ and $t_1 \cdots t_k$ to
  $U$. Let $P$ be a set of paths minimizing the edges outside of
  $K$. Label the non-ending vertices as $u_i$. Let $s'_i$ denote the
  end the path $P_i$ starting at $s_i$ in $U$ and $t'_i$ denote the
  end of $t_i$ in $U$. Note that $s'_i u_i$ in $K$ can only contain
  vertices on $P_i$, by minimality of $P$. Thus there exists a path
  from $s_i$ to $u_i$ that is disjoint from all other paths in
  $P$. Similarly there is a path from $t_i$ to $u_i$, show that $G$ is
  $k$-linked.
\end{proof}

\begin{theorem} (4.2.6) In a two connected plane graph, every face is
  bounded by a cycle.
\end{theorem}
\begin{proof} Let $G$ be a two connected plane graph of minimum order
  which contradicts the proposition. Since $G$ is 2-connected, it can
  be constructed by adding $H$-paths, thus there exists graph $H$ and
  an $H$-path $P$ such that $G = H \cup P$. By the minimality of $G$,
  every face in $H$ is bounded by a cycles, so $V(P) \subseteq
  F$. Since $P$ only effects $1$ face in $G$, we may assume that
  $H = K_3$. There is obviously no way connect an $H$-path to $K_3$
  invalidating the theorem. Thus theorem is true.
\end{proof}


\begin{theorem} (4.2.7) The face boundaries in a $3$-connected plane
  graph are precisely its non-separating induced cycles.
\end{theorem}
\begin{proof} Let $C$ be a non-separating induced cycle, then $C$ must
  obviously bound a face of $G$. For the converse. Let $f$ be a face
  in $G$ and let $C$ be the cycle bounding $f$. If $C$ has a chord
  $xy$, then the components of $C-\{x, y\}$ are linked by a $C$ path
  in $G$, because $G$ is $3$ connected. This path must run through
  $f$, a contradiction. Thus $C$ is induced. Now it remains to show
  that $C$ does not separate any two vertices in $G - C$. By Menger's
  theorem, there exists $3$ vertex independent paths between any two
  vertices $x$ and $y$. Clearly $f$ lies inside a face of the union of
  these paths. This path is bounded by only two of the paths, thus the
  third path must avoid $C$ entirely.
\end{proof}

\begin{theorem} (4.2.8) A plane graph of order at least 3 is maximally
  planar iff it s a plane triangulation.
\end{theorem}
\begin{proof} Let $G$ be a plane graph of order at least 3. Obviously
  any plane triangulation must be maximally planar, since any edge
  added must be added within a face. To show the other direction, let
  $G$ by maximally planar and pick a face $F$. Note that $G[F]$ must
  be complete, otherwise we could add an edge between the two
  unconnected elements in $F$ through $F$ to find a larger planar
  graph. Since $F$ is planar, this implies that $|F| \leq 3$. If
  $|F| < 3$, then there exists only one face in $G[F]$, if there is an
  edge from $F$ to another vertex, that vertex must also be in $F$
  because there is now way for that edge to close off $F$ from a new
  space. Thus if $|F| < 3$ there must be no other edges in the graph
  contradicting $|G| \geq 3$. Thus $G[F] = K_3$, proving the claim.
\end{proof}

\begin{theorem} (4.2.9) Let $G$ be a connected planar graph with $n$
  vertices, $m$ edges, and $f$ faces. Then $n - m + f = 2$.
\end{theorem}
\begin{proof} The proof is by induction on $|m|$. For the base case,
  consider $n = m + 1$, in which case $G$ must be a tree since $G$ is
  connected. Thus there is exactly one face yielding the desired
  inequality. Otherwise $m > n - 1$. Then $G$ must contain a
  cycle. Let $C$ be a cycle of minimum length and delete an edge from
  the cycle, to obtain a graph $G'$. $G'$ contains exactly 1 fewer
  edges and 1 fewer faces thus $n - (m - 1) + (f - 1) = 2$ which
  immediately implies out conclusion.
\end{proof}

\begin{theorem} (4.4.2) A graph contains $K^5$ or $K_{3,3}$ as a minor
  iff it contains $K^5$ or $K_{3,3}$ as a topological minor.
\end{theorem}
\begin{proof} The reverse direction is obvious since all topological
  minors are also minors. To show that forward direction, if $G$
  contains a $K_{3,3}$, it obviously contains a $TK_{3,3}$. So assume
  $G$ contains a $K^5$ minor. Let $K$ be a minimal $K^5$ minor in
  $G$. Then every branch set of $K$ induces a tree in $K$, and between
  any two branch sets, $K$ has exactly one edge. Let $T_x$ be the tree
  on vertex set $V_x$, including the edges connecting $V_x$ to the
  other branch sets. From minimality, there are exactly $4$ such
  edges, and $4$ leafs in $T_x$. If each of the $5$ treas from a
  $TK{1,4}$, then we have found a $TK_5$. If $T_x$ is not a
  $TK_{1,4}$, then it has exactly of vertices of degree
  $3$. Contracting $V_x$ into these vertices and the other branch sets
  into a vertex yields a graph on $6$ vertices containing a
  $TK_{3,3}$, Thus $G$ contains a $TK_{3,3}$.
\end{proof}

\begin{theorem} (4.4.3) Every $3$-connected graph without a $K_5$ or
  $K_{3,3}$ topological minor is planar.
\end{theorem}
\begin{proof} Proof is by induction. Note this applies when
  $|G| \leq 4$. Otherwise, let $G$ be a planar graph on at least $5$
  vertices. Since $G$ is $3$-connected, there exists an edge $xy$ such
  that $G'=G/xy$ is $3$-connected. By the inductive hypothesis this
  graph has a planar drawing. Consider the face $f$ containing
  $v_{xy}$ in $G'-v_{xy}$, and let $C$ be the boundary of $f$. Note
  that $C$ must be a cycle since $G'$ must be $3$-connected. Let
  $X = \{ x_1, \cdots x_k\}$ be $N(x) \cap V(C)$ where the vertices
  are enumerated in order around $C$. Define $Y$ to be
  $N(y) \cap V(C)$. Let $P_i$ be the path on $C$ from $x_i$ to
  $x_{i + 1}$.

  Note that $N(y) \subseteq P_i$ for some $i$. Assume to the contrary,
  then there exists vertices $y'$ and $y''$ not contained in the same
  $P_i$, and these two vertices are separated by two vertices $x'$ and
  $x''$ on $C$. Pick $y'$, $y''$, $x'$, $x''$ such that the number of
  distinct vertices in maximized. If every vertex is distinct, then we
  get a $TK_{3,3}$ with partition $(\{x, y', y''\}, \{y, x',
  x''\})$. Because $G$ is $3$-connected, the fewest distinct vertices
  possible is $3$, and this is only achieved if $N(X) = N(Y)$ with
  $|N(X)| = 3$. In this case though, $N(X)$ along with $x$ and $y$
  form a copy of $TK^5$. Thus $N(Y) \subseteq V(P_i)$ for some
  $i$. But then we can construct a planar drawing of $G$ by starting
  with a planar drawing of $G-y$ and placing $y$ and its edges inside
  the face containing $x$ and $P_i$.
\end{proof}

\begin{theorem} (4.4.6) If $G$ is a graph, then the following are
  equivalent:
  \begin{itemize}
    % \item $G$ is planar
  \item $G$ contains no $K_5$ or $K_{3,3}$ minor.
  \item $G$ contains not topological $K_5$ or $K_{3,3}$ minor.
  \end{itemize}
\end{theorem}
\begin{proof} The proof immediately follows from 4.4.3 and 4.4.2, and
  the observation that all maximally planar graphs on at least $4$
  vertices are $3$-connected.
\end{proof}

\begin{theorem} (5.1.2) Every planar graph is $5$-colorable.
\end{theorem}
\begin{proof} The proof is the proof that every planar graph is 5
  chooseable (5.4.2).
\end{proof}

\begin{theorem} (5.2.4 Brooks Theorem) Let $G$ be connected graph. If
  $G$ is neither complete, nor an odd cycle, then
  $\chi(G) \leq \Delta(G)$.
\end{theorem}
\begin{proof} Note that the statement is obviously true if
  $\Delta(G) \leq 2$ since in that case $G$ is either a path or a
  cycle. Let $G$ be a graph minimizing $|G|$ which is a counter
  example to the theorem. Note that $G$ is not a complete graph. Pick
  any vertex $x \in G$. Now consider the graph $G-x$ and note that
  $\chi(G-x) \leq \Delta(G)$. This follows since if $G - x$ is a
  complete graph or odd cycle, then $\Delta(G -x) < \Delta(G)$ since
  $G$ is connected and otherwise $\Delta(G - x) \leq \Delta(G)$. Color
  $G - x$ with $\Delta(G)$ colors. Since $G-x$ cannot be colored with
  $\Delta(G)$ colors, the $N(x)$ must contain at least $\Delta(G)$
  colors. Let $N(x) = \{v_1, v_2, \cdots v_{\Delta(G)} \}$ where $v_i$
  has color $i$ in $f$.

  Define $H_{i,j}$ be component of the graph induced by vertices with
  colors $i$ and $j$ in the coloring of $G - x$. Note that $v_i$ and
  $v_j$ must appear in the same component in $C_{i,j}$, otherwise, in
  one of the components we could swap the colors to obtain a coloring
  $f'$ in which $v$ and $w$ have the same color, allowing us to color
  $x$ with either color $i$ or $j$.

  We will show that $C_{i,j}$ must be a path. Otherwise there exists a
  vertex $u$ which is as close to $v$ as possible which has degree at
  least $3$. Then in $G-x$, $N(u)$ can contain at most $\Delta(G) - 1$
  colors since 3 of the vertices have at most $2$ colors. Thus we can
  swap the color of $u$ to the missing color to create a coloring
  $f'$. Since $u$ was the color closest to $v$, in $f'$ $C_{i,j}$ $v$
  and $w$ must now be in different components (if both $u$ and $v$ are
  even in $C_{i,j}$ anymore), which will allow us to color $x$ once
  again. Thus $C_{i,j}$ must be a path.

  Now consider two such paths $P_{i,j}$ and $P_{j,k}$ between vertices
  $v_i$, $v_j$ and $v_j$ and $v_k$. Then
  $V(P_{i, j}) \cap V(P_{j,k}) = v_j$. Otherwise there exists a vertex
  $u \in V(P_{i, j}) \cap V(P_{j,k})$. The vertex $u$ satisfies that
  $4$ of its neighbors get $2$ colors. Thus $N(u)$ contains at most
  $\Delta(u) - 2$ colors. Thus $u$ could be recolored creating the
  same situation as described previously.

  Now finally we will use the previous properties to construct a
  coloring of $G$. Note that $G$ must not be a complete graph, thus
  there exists a vertex $x$ who's neighborhood is not complete. Let
  $f$ be a coloring of $G-x$ such that $v_1v_2$ is not an edge in
  $G$. Now construct a new coloring $f'$ by swapping the colors $1$
  and $3$ on $P_{1,3}$. Consider the vertex $u$ which is the neighbor
  of $v_1$ on $P_{1,2}$. Since no vertices of color $2$ were changed,
  $u$ must be the neighbor of $v_1$ in $P'_{2, 3}$ or else we are
  finished by the above properties. Also, since
  $P_{1,3} \cap P_{1,2} = v_1$, $u$ must be in $P'_{1,2}$ as well or
  we are finished by the above properties again.  Finally, note that
  $u$ contradicts the final property allowing us to construct a proper
  coloring anyways.
\end{proof}

\begin{theorem} (Mycelskis construction) There exists a triangle free
  graph with chromatic number $k$ for any integer $k$.
\end{theorem}
\begin{proof} The proof is by induction. For $k \leq 2$, the problem
  is trivial.  Let $G$ be a $k$ chromatic triangle free
  graph. Construct a new graph $G'$ on $V(G)$, $U$ and $\{w\}$, where
  $G'[V(G)] = G$, $v_i u_j$ is an edge iff $v_i v_j$ is an edge in
  $G$, and where $N(w) = U$. Note that after doing this, $G'$ is still
  triangle free. Note that a coloring of $V(G)$ can be extended to $U$
  to obtain a proper coloring of $G'$. Thus
  $\chi(G') \leq \chi(G) + 1$. Now it suffices to show that
  $\chi(G) < \chi(G')$. Let $G$ be a proper color of $G'$ on fewer
  than $\chi(G) + 1$ colors. We may assume that $G(w) = k$. Thus there
  are $k-1$ colors on $U$. Let $A$ be the set of vertices on $G$ with
  color $k$. For each vertex in $v_i \in A$, change the color of $v_i$
  to $u_i$. Since $v_i$ and $u_i$ the same neighbors (excluding $w$),
  this gives a proper coloring of $G$ on fewer than $\chi(G)$ colors,
  a contradiction.
\end{proof}

\begin{definition} A graph $G$ is $k$-constructible if it is
  isomorphic to one of the following:
  \begin{itemize}
  \item[(i)] $K^k$.
  \item[(ii)] $(G' + xy)/xy$ where $G'$ is a $k$-constructible graph
    with vertices $x$ and $y$ that are non-adjacent.
  \item[(iii)] $(G_1 \cup G_2)-xy_1 - xy_2 + y_1y_2$ where $G_1$ and
    $G_2$ are $k$-constructible with $V(G_1) \cap V(G_2) = x$ and
    $xy_i \in G_i$.
  \end{itemize}
\end{definition}
\begin{theorem} (5.2.6) Let $G$ be a graph $k \in \mathbb{N}$. Then
  $\chi(G) \geq k$ iff $G$ has a $k$-constructible subgraph.
\end{theorem}
\begin{proof} The proof of the forward direction is done by induction
  on the minimum number of steps to construct $G$ using the
  $k$-constructible constructions. Let $G$ be a $k$-constructible
  graph. If $G = K^k$, then obviously $\chi(G) \geq k$. Otherwise if
  the last step to construct $G$ was such that $G = G' + xy /xy$, then
  a coloring of $G'$ immediately induces a color on $G$ by setting the
  color of $x$ and $y$ to be the same as the contracted vertex and
  leaving all other vertices the same color. Finally, if the last step
  to construct $G$ results form $G = G_1 + G_2 -xy_1 - xy_2 + y_1y_2$,
  then a coloring of $G$ forces the color of $x$ to be 1different from
  one of $y_1$ or $y_2$. Thus the coloring of $G$ induces a coloring
  on $G_1$ or $G_2$, thus $\chi(G) \geq k$.

  For the reverse direction, let $G$ be an edge maximal graph with
  $\chi(G) = k$ without a $k$-constructible subgraph. Note $G$ is not
  complete $r$-partite, because otherwise $G$ would have to contain
  $K^k$. Thus there exists an edge $y_1 y_2$ and vertex $x$ such that
  $xy_1$ and $xy_2$ is not an edges. Note $G + x y_i$ must contain a
  $k$-constructible subgraph $H_i$ because $G$ is edge maximal. Let
  $H'_2$ be an isomorphic copy of $H_2$ such that
  $H'_2 \cap H_1 = \{x\}$, $|V(H'_2) \cap V(H_2)|$ is maximized, and
  all vertices in $H'_2 - H_2$ are not in $G$. Then
  $H_1 \cup H'_2 - xy_1 - xy'_2 + y_1y'_2$ is $k$-constructible. Now
  for every vertex $v' \in H'_2 - G$, let $v$ be the corresponding
  vertex in $H_2$ construct another $k$-constructible graph by adding
  and then contracting the edge $v'v$ to create another
  $k$-constructible graph. This new graph $H$ is then a
  $k$-constructible subgraph of $G$, a contradiction.
\end{proof}

\begin{theorem} (5.3.2 Vizing's Theorem) Every graph $G$ satisfies
  $\Delta(G) \leq \chi'(G) \leq \Delta(G) + 1$.
\end{theorem}
\begin{proof} Note that obviously $\Delta(G) \leq \chi'(G)$. So
  consider the inequality of $\chi'(G) \leq \Delta(G) + 1$. This
  inequality is true when $G$ is the empty graph. Let $G$ be the
  counter example minimizing $||G||$. Note that for any proper edge
  coloring of $G-xy'$ with $xy' \in G$, there must be some color
  $\alpha$ missing among edges ending at $x$ and some color $\beta$
  missing among edges ending at $y'$. Now consider a maximal
  $\alpha \beta$ path $P$ starting at $y'$. Then this path must end at
  $x$, otherwise we could swap the coloring on the path $P$ and color
  the edge $xy'$ with $\alpha$ to get a proper coloring of $G$.  Let
  $x$ be a vertex in $G$, and $xy_0$ an edge. Then $G-xy_0$ has an
  edge coloring $f_0$ using at most $\Delta(G) + 1$ colors. Let
  $y_0, y_1, \cdots y_k$ be a maximal sequence of the neighbors of $x$
  such that the color of the edge $x y_{i + 1}$ is missing among edges
  ending at $y_i$. Note that for any edge $x y_j$ we can create a
  coloring $f_j$ for $G-x y_j$ based on the coloring $f_0$ by
  assigning $x y_i$ the color of $x y_{i + 1}$ in $f$ for $i < j$ and
  keeping all other colors the same. Let $\beta$ be the coloring
  missing at $y_k$. Then the maximal $\alpha$-$\beta$ path must start
  at $y_k$ and end at $x$. Let $y$ be the vertex preceding $x$ on this
  path. From the maximality of the sequence $y_0, y_1, \cdots y_k$,
  $y = y_j$ for $j < k$ since $x y$ has color $\beta$. Now consider
  the coloring $f_j$ of $G - xy$. Then the $\alpha-\beta$ path
  starting at $y_k$ must end at $y$. Thus the colors on this path
  could be swapped so that $xy_j$ can be colored with $\alpha$ to
  obtain a proper edge coloring of $G$.
\end{proof}

\begin{theorem} (5.4.2) Every planar graph is $5$-chooseable.
\end{theorem}
\begin{proof} Proof by induction, but with the extra hypothesis that
  every connected planar graph whose external vertices form a cycle
  can be list colored if the list on all interior vertices has size at
  least $5$ and all exterior vertices has size at least $3$ and two
  adjacent exterior vertices $v_1$ and $v_2$ are already colored. This
  is sufficient because one can always add edges to a planer graph $G$
  to form a cycle, unless $|G| < 3$ in which case the claim is
  trivially true.

  For the base case, note the new claim trivially holds when
  $|G| \leq 3$. Now let $G$ be a connected planar graph with exterior
  vertices in a cycle $C = v_1 \cdots v_k$ and assign $G$ a lists such
  that it satisfies the inductive hypothesis. To begin with, assume
  that $G$ has a cord $vw$. Then we can split $G$ into two planar
  graphs, one of which contains $v_1$ and $v_2$ and the other which
  doesn't. By induction we can color the graph containing $v_1$ and
  $v_2$. Now setting the colors of $v$ and $w$ as found in the last
  coloring we can again by our inductive hypothesis color the other
  cycle. Otherwise for the vertex $v_k$, reserve two colors which are
  not the color of $v_1$ and delete these colors from its interior
  neighbors. Then by the inductive hypothesis we can color
  $G-v_k$. Since $v_{k-1}$ is the only neighbor of $v_k$ which could
  have one of the colors reserved for $v_k$ we can still color $v_k$.
\end{proof}

\begin{theorem} (5.4.3) Let $H$ be a graph and ${(S_v)}_{v \in V(H)}$
  a family of lists. If $H$ has an orientation $D$ with
  $d^+(v) < |S_v|$ for every $v$, and such that every induced subgraph
  of $D$ has a kernel, then $H$ can be colored from the lists $S_v$.
\end{theorem}
\begin{proof} The proof is by induction on $|G|$. If $|G| = 0$, then
  the claim is obviously true. Otherwise let $G$ be a graph satisfying
  the hypothesis. Let $\alpha$ be a color and let $T$ be the set of
  vertices such that the color $\alpha$ is a possible color. Then the
  graph $G[T]$ has a kernel $U$. Coloring all the vertices of $U$ with
  the color $\alpha$ and deleting $U$ from $G$ and $\alpha$ from the
  list of all other vertices in $T$ then creates a smaller graph still
  satisfying the inductive hypothesis, thus completing the proof by
  induction.
\end{proof}

\begin{theorem} (5.4.4) Every bipartite graph $G$ satisfies
  $ch'(G) = \chi'(G)$.
\end{theorem}
\begin{proof} Let $G$ be a bipartite graph with partition $X,Y$ and
  let $c$ be an edge coloring of $G$ in $\chi'(G)$ colors. Place an
  ordering on the colors and let $H$ be the line graph of $G$ with a
  direction applied to the edges as follows: Let $e_1$ and $e_2$ be
  intersecting edges with $c(e_1) < c(e_2)$. If $e_1e_2$ meet in $x$
  direct the edge from $e_1$ to $e_2$ and if $e_1e_2$ meet in $Y$,
  direct the edge from $e_2$ to $e_1$. Now for a given edge $e$, the
  out edges of $e$ that meet in $y$ must have colors strictly larger
  than $c(e)$, and the color for edges meeting in $X$ must have color
  strictly less than $c(e)$. Since any two neighbors of $e$ meeting in
  $X$ or $Y$ are also neighbors, we get that $d^+(e) < k$. Now it
  suffices to show that every induced subgraph of $H$ contains a
  kernel. Let $Z$ be a subset of $X \cup Y$ and consider $G[Z]$. Let
  the edge colors represent preferences. Then we can find a maximum
  stable matching given these preferences. This stable matching is an
  independent set $S$ in $H$. Since it is maximum, all out edges of a
  vertex in $H$ must enter $S$, otherwise we would get a larger stable
  matching. Thus $S$ is the desired kernel, and by the previous lemma,
  $ch'(G) = \chi'(G)$.
\end{proof}

\begin{theorem} (5.5.1) A chordal graph can be constructed recursively
  by pasting along complete subgraphs.
\end{theorem}
\begin{proof} Consider a chordal graph $G$. If $G$ is complete, then
  we are done, otherwise let $a$ and $b$ two unconnected vertices. Let
  $X$ be a minimal $a$-$b$ separator. Note that for any two vertices
  $u, v \in X$, there must be a path through the component of $G - X$
  containing $a$ and similarly in the component containing $b$. This
  creates a cycle of length at least $4$, thus $u,v$ must be an edge
  since $G$ is chordal.
\end{proof}

\begin{theorem} Every chordal graph contains a simplicial vertex
\end{theorem}
\begin{proof} The proof is by induction. Note this is true if
  $G = K^1$. Now let $x \in V(G)$ such that $G-x$ is connected. Such a
  vertex must as exist, otherwise consider a vertex $x$ such that
  $G-x$ contains a component $C$ of minimum size. Then for $x' \in C$,
  $G-x'$ must contain a smaller component as $x'$ is a cut vertex by
  assumption. If $N(x) = V(G) - x$, then $G-x$ has a simplicial vertex
  and adding $x$ to $G$ does not change this, so we are done
  again. Otherwise $N(x)$ is a minimal vertex cut separating $x$ from
  the rest of the graph. Since $G - x$ is connected, there is exactly
  one component $C$ in $G - x - N(x)$. Also, by minimality there must
  be a $u,v$ $C$-path for all vertex pairs $u,v \in N(x)$. Thus $u,v$
  must be an edge since $G$ is chordal. Therefore $x$ is a simplicial
  vertex and we are done.
\end{proof}

\begin{theorem} (5.5.2) All chordal graphs are perfect
\end{theorem}
\begin{proof} This proof is by induction. If $|G| = 1$. Otherwise,
  every induced subgraph of $G$ is chordal, so we are done by
  induction. Thus it just suffices to consider if
  $\chi(G) \leq \omega(G)$. Since $G$ is chordal, $G$ can be created
  by pasting along a complete subgraph. We can independently color
  these two chordal subgraphs, permuting colors if necessary to create
  a coloring of $G$. Since the subgraphs are chordal we get that
  $\chi(G) \leq \omega(G)$ by induction.
\end{proof}

\begin{theorem} (5.5.4 Lovasz) A graph $G$ is perfect iff $\bar{G}$ is
  perfect.
\end{theorem}
\begin{proof} This proof is by induction. Note that $|G| = 1$, then
  this is true. Now let $G$ be a graph with $|G| > 1$. By inductions,
  for any strict subgraph of $G$, the complement of that subgraph is
  perfect, thus from symmetry, all that we need to show is that if $G$
  is perfect, then $\chi(\bar{G}) \leq \omega(\bar{G})$. Let
  $\mathcal{A}$ be the set of independent sets of size
  $\alpha(G) := \alpha$. If $G$ contains a complete subgraph $X$ such
  that $X \cap A \neq \emptyset$ for all $A \in \mathcal{A}$, then
  $\chi(\bar{G}) \leq \chi(\bar{G - X}) + 1 = \omega(\bar{G - X}) + 1
  = \omega(\bar{G})$. Thus $\bar{G}$ is perfect. Otherwise there is no
  such $X$, in which case we will show that no such graph can
  exists. Otherwise, let $\mathcal{K}$ represent the set of all
  complete subgraphs of $G$. Then for each $K \in \mathcal{K}$, there
  exists an independent set $A_K \in \mathcal{A}$ such that
  $K \cap A_K = \emptyset$. Create a new graph $G'$ by expanding each
  vertex $x$, $k(x)$ times where $k(x) = |{K | x \in A_K}|$. Note that
  $G'$ is still perfect since expanding a vertex creates a graph which
  is still perfect. Now in $G'$ we get that there is an
  $X \in \mathcal{K}$ such that
  $\omega(G') = \sum_{x \in X}, k(X) < |\mathcal{K}|$ since $A_K$ is
  independent so each set $K$ can cause at most one vertex of $X$ to
  expand, and $X \cap A_X = \emptyset$. On the other hand we get that
  $|G'| \geq |K| \alpha$ since each $K \in \mathcal{K}$ got its own
  copy of $A_K$ in the vertex expansion. Note the expansion didn't
  change the size of the maximum independent set, so
  $\chi(G') \geq |K|$, contradicting the perfection of $G'$.
\end{proof}

\begin{theorem} (5.5.5) Any graph obtained from a perfect graph by
  expanding a vertex is again perfect.
\end{theorem}
\begin{proof} This proof is by induction on perfect graphs. If
  $G = K^1$, then we are done. Let $G$ be any larger perfect
  graph. Let $G'$ denote $G$ with the expanded vertex $x$. By
  induction, any subgraph of $G'$ satisfies that
  $\chi(G') \leq \omega(G')$, thus is suffices to consider only
  $G'$. If $x \in K^{\omega(G)}$, then $G'$ contains a
  $K^{\omega(G) + 1}$, so we can trivially color $x'$ and are
  done. Otherwise let $\alpha$ be the color of $x$ in minimum coloring
  of $G$. Let $X$ be the set of vertices in $G-x$ with color
  $\alpha$. Note that every $K^{\omega(G)}$ must contain a vertex in
  $X$. Thus $G - X$ is a perfect graph with
  $\omega(G - X) = K^{\omega(G) - 1}$. Thus $G - X$ can be colored
  with $\omega(G) - 1$ colors. Also, $X + x'$ is an independent set,
  thus $G'$ can be colored with $\omega(G')$ colors.
\end{proof}

\begin{theorem} (5.5.6) A graph $G$ is perfect iff
  $|H| \leq \alpha(H) \omega(H)$ for all induced subgraphs
  $H \subseteq G$.
\end{theorem}
\begin{proof} For the forward direction, assume that $G$ is
  perfect. Let $f$ be a coloring of $H$ for some subgraph of $G$. Then
  $|H| \leq \alpha(H) \chi(H) = \alpha(H) \omega(H)$.

  To show the reverse direction, let $G$ be the minimal counter
  example and note that $G$ is non-empty. Then every subgraph of $G$
  must be perfect. Thus for every independent set $U$,
  $\chi(G - U) = \omega(G - U) \leq \omega(G)$ because $G - U$ is
  perfect and has a coloring of size $\omega(G - U)$. In particular,
  since $G$ is not perfect by assumption, we get that
  $\omega(G - U) = \omega(G)$ implying there exists a $K^\omega$ in
  $G-U$ for any independent set $U$. Let
  $A_0 = \{ u_1 \cdots u_{\alpha(G)}$ be a set of of independent
  vertices in $G$ of size $\alpha(G)$. For each $u_i \in A$, consider
  an $\omega(G)$ coloring of $G - u_i$ and let $A_{i}$ through
  $A_{i + \omega(G)}$ denote these color classes. For each independent
  set $A_j$, there must exists a $K^{\omega(G)}$ in $G - A_j$. Let
  $K_j$ denote the vertices of one such copy of $K^{\omega(G)}$ in
  $G - A_j$. Construct a $\alpha(G)\omega(G) + 1 \times n$ matrix $A$
  where the $j$th row forms an incidence vector of the vertices of
  $A_j$ with the vertices of $G$. Similarly construct a
  $n \times \alpha(G)\omega(G) + 1$ matrix $B$ where the $j$th row is
  the incidence vector of the vertices of $K_j$ with the vertices of
  $G$. Consider the matrix $C = AB$. By the construction, element
  $c_{ij}$ of $C$ has the value of $|A_i \cap K_j|$, which must be
  equal to $1$ if $i \neq j$ and $0$ if $i = j$. Thus the
  $\alpha(G)\omega(G) + 1 \times \alpha(G)\omega(G) + 1$ matrix $C$ is
  invertible, implying the rank of $A$ is at least
  $\alpha(G) \omega(G) + 1$ implying
  $|G| \geq \alpha(G) \omega(G) + 1$, contradicting the assumption
  that $|G| \leq \alpha(G) \omega(G)$.
\end{proof}

\begin{theorem} (6.1.1) If $f$ is a circulation, then
  $f(X, \bar{X}) = 0$ for every set $X \subseteq V$.
\end{theorem}
\begin{proof} $f(X, \bar{X}) = f(X, V) - f(X,X) = 0 - 0 = 0$.
\end{proof}

\begin{theorem} (6.2.2 Max flow / Min Cut) In every network, the
  maximum total flow equals the minimum capacity of a cut.
\end{theorem}
\begin{proof} It is obvious that a max flow must be smaller than a
  minimum capacity cut. To show the other direction constructed
  through a maximal sequence of flows $f_0 \cdots f_n$ where $f_0$
  gives a flow of 0 to every edge. For each $f_i$ associate a set
  $S_n$ which is the set of vertices which can be reached by a walk
  $W$ that only uses $e$ with $f(e) < c(e)$. If $t \in S_i$, then we
  construct $f_{i + 1}$ by adding the maximum possible flow along a
  walk $W$. Since this increases the flow by a fixed amount, and the
  flow is bounded by the minimum cut. This sequence must end at some
  $n$. But than $(S_n, \bar{S_n})$ forms a cut, which by definition
  has max flow along each edge. Thus proving the theorem.
\end{proof}

\begin{theorem} (6.3.1) For every multigraph $G$, there exists a
  polynomial $P$ such that for every finite Abelian group $H$, the
  number of nowhere zero $H$-flows on $G$ is $P(|H| - 1)$.
\end{theorem}
\begin{proof} The proof will be by induction on $|G|$ and $||G||$. To
  begin with, note that if $G$ is a graph with only loops, then we can
  obviously generate such a polynomial. Otherwise there must exist
  some edge $xy$ with $x \neq y$. Let $P_1$ be the polynomial for the
  number of flows in $G-e$ and $P_2$ for $G/e$. Then we will show that
  the polynomial $P$ for $G$ is $P= P_2 - P_1$. Note the flows in
  $G-e$ are precisely the nowhere zero flows on every edge except for
  $e$. Thus it suffices to show that $P_2$ is the number of flows on
  $G$ and the nowhere zero flows on every edge except $e$. To show
  that this works, consider a flow on $G$ that is possibly only zero
  on $e$. Since $G/e$ doesn't destroy any edges, this flow translates
  directly to a flow in $G/e$. Now it suffices to show there is an
  injection from the flows on $G/e$ to $G$. Let $g$ be any flow on
  $G/e$. Now consider the image of this flow in $G$ where each edge
  that is not $xy$ is given the same value as in $G/e$. Then only the
  vertices $x$ and $y$ have imbalances in flow. But the conservation
  of flow then implies that every amount of flow accumulating at $x$,
  a similar amount must be lost from $y$ (and vice versa). Thus there
  is exactly on weighting of the edge $e$ such that we created a flow
  $f$ in $G$ corresponding to $g$, thus completing the proof.
\end{proof}

\begin{theorem} (6.3.3) A multigraph admits a $k$-flow iff it admits a
  $\mathbb{Z}_k$ flow.
\end{theorem}
\begin{proof} Let $\sigma_k$ be the obvious projection from
  $\mathbb{Z}$ to $\mathbb{Z}_k$ If $f$ is a $k$ flow then $\sigma f$
  is obviously a a $\mathbb{Z_k}$ flow. Now consider the other
  direction. Let $g$ be a $\mathbb{Z}_k$ flow. Let $F$ be the set
  functions such that $|f(e)| \leq k$, $f(xy) = - f(yx)$ for all
  $f \in F$. Note that an element of $F$ can be constructed in the
  obvious way from $g$, thus $F$ is non-empty. We would like to show
  that there exits an element of $F$ which is flow. Towards this end,
  consider the error of a flow to be $K(f) = \sum(|f(v, V)|)$. If
  there exists a function $f$ such that $K(f) = 0$, then we are
  done. Otherwise, let $f$ be such that $K(f)$ is minimum. Since
  $K(f) > 0$, there exists a vertex $v$ such that $f(v,V) > 0$. Let
  $A$ be the set of vertices which can be reached from $f$ by a
  positive walk. $K(f)$ restricted to the vertices in $A$ has a value
  of $f(A, \bar A - x) + f(A,A) + f(A,v)$. By the definition,
  $F(A, \bar A - x) <= 0$, $f(A,v) \leq 0$. Also $f(A, A) = 0$ since
  each edge is double counted, but in opposite directions so the flow
  cancels. Thus there is a vertex such that $f(v', V)$ is
  negative. Taking a positive walk from $v$ to $v'$, we can reverse
  that walk, resulting in an $f'$ with smaller error. Thus $K(f) = 0$
  providing a flow.
\end{proof}

\begin{theorem} (6.5.3) For every dual pair $G$, $G^*$ of plane
  multigraphs, $\chi(G) = \phi(G^*)$.
\end{theorem}
\begin{proof} To begin with, note that if $G < 3$, then the claim is
  obvious. The proof will be by induction on the number of bridge
  edges in $G$. Skipping the base case for the moment, assume that $G$
  contains a bridge edge $e$, $|G| \geq 3$, and the inductive
  hypothesis holds. Note that $e^*$ is a loop. But then we get that
  \[
    \chi(G) = \chi(G/e) = \phi(G^* - e^*) = \phi(G^*)
  \]
  where the first equality comes from the fact that $G - e$ contains
  two components that can be independently colored, the second follows
  from the inductive hypothesis, and the third from the fact that
  $e^*$ is a loop and $G^* - e^*$ contains an edge.

  Now to complete the proof it suffices to show that the claim holds
  in the base case where $G$ contains no bridge edges. To begin with,
  note that if $G$ has a loop, then $G^*$ has a bridge, in which case
  $\chi(G) = \infty = \phi(G^*)$ by convention.

  Assume that there $G$ has no loop. Then it is known that Then there
  is a bijective mapping from an orientation $E^*$ to an orientation
  of $E$ such that a flow $g$ in $G^*$ is a circulation iff the flow
  $f$ produced by composing the bijection between $E$ and $E^*$ and
  $g$, satisfies that $f(X,X) = 0$ and $f(C) = 0$ for any directed
  cycle under the orientation of $E$.

  Assume that $\phi(G^*) = k$. Let $g$ be a $Z_k$ circulation on $G^*$
  and $f$ the flow in $G$ from above. Consider a normal spanning
  subtree $T$ with root $r$ of $G$ with the orientation used to create
  $F$. Construct a coloring $c$ by coloring the vertex $v \in T$ with
  the flow along the path from $r$ to $v$. Then this flow creates a
  proper coloring of $G$. Let $e = xy$ be an edge in $G$. If $e = xy$
  is an oriented edge in $T$, then since $g$ is a flow we get that
  $c(x) \neq c(y)$. Otherwise, the edge $e$ completes a directed cycle
  $C$ in the orientation of $G$ with $f(C) = 0$. Since $g$ is a
  circulation we once again get that $c(x) \neq c(y)$. Thus
  $\chi(G) \leq k$.

  Finally assume that $\chi(G) = k$. Let $c$ be a $k$ coloring of
  $G$. Define $f(u,v) = c(u) - c(v)$ for an directed edge $uv$. Note
  that by definition $f(X,X) = 0$ and $f(C) = 0$ for any directed
  cycle $C$. Thus there exists a flow $g$ in $G^*$ which is a
  circulation. Thus $\phi(G^*) \leq k$, completing the proof.
\end{proof}

\begin{theorem} (6.4.3) For all even $n \geq 6$, $\phi(K^n) = 3$.
\end{theorem}
\begin{proof} To begin with, note that $\phi(K^n) > 2$ because every
  vertex has odd degree when $n$ is even. The proof of the upper bound
  will by by induction on $n$. For the base case consider $n = 6$ with
  $G = K^6$. Then the edges of $G$ can be partitioned to form $G_1$,
  $G_2$ and $G_3$ where $G_1$ and $G_2$ are copies of $K^3$ and $G_3$
  is a copy of $K_{3,3}$.Note that $K_3$ has a $2$-flow and $K_{3,3}$
  has a $3$-flow obtainable by assigning a flow of $1$ to every edge
  from $A$ to $B$ where $(A,B)$ is the bipartition of
  $K_{3,3}$. Combining all these flows yields a $3$-flow on $G$.

  Now fix $n \geq 8$ and assume the inductive hypothesis holds for all
  smaller even $n$. Note that the edges of $G = K^n$ can be
  partitioned to form $3$ graphs $G_1$, $G_2$, and $G_3$, where
  $G_1 = K^{n - 2}$, $G_2 = K_{2, n - 4}$, and $K^4 - e$ for any edge
  $e \in K^4$. By the inductive hypothesis $G_1$ contains a
  $3$-flow. For $G_2$, every vertex has even degree, and thus $G_2$
  contains a $2$-flow. Finally for $G_3$, let $v_1$, $v_2$, $v_3$, and
  $v_4$ denote the vertices of $G_3$ where the edge $v_3 v_4$ is the
  only non-edge in $G_3$. Letting the $3$-flow $f:Z^k \to E(G_3)$
  defined by setting $f(v_i v_1) = 1$ and $f(v_i v_2) = 2$. Note that
  that $f(v_1v_2) = -f(v2 v1)$ so this definition is consistent. Since
  $v_1$ and $v_2$ have degree $3$ and all entering edges have the same
  flow value, no flow accumulates on $v_1$ or $v_2$. Similarly for
  $v_3$ and $v_4$, since the two edges entering have flow $1$ and $2$
  no flow accumulates at $v_3$ and $v_4$. Thus $G_3$ has a
  $3$-flow. Combining the flows on $G_1$, $G_2$, and $G_3$ yields the
  desired $3$-flow on $G$.
\end{proof}

\begin{theorem} (6.6.1) Every bridgeless graph has a $6$-flow.
\end{theorem}
\begin{proof}
\end{proof}

\begin{theorem} (9.1.1) For every $r \in \mathbb{N}$ there exists an
  $n \in \mathbb{N}$ such that every graph of order at least $n$
  contains either $K^r$ or $\bar{K^r}$ as an induced subgraph.
\end{theorem}
\begin{proof} Let $R(s,t)$ be the least $n$ such that every graph on
  $n$ vertices contains either $K^s$ or $\bar{K^t}$ as an induced
  subgraph. Note that $R(1,t)$ and $R(s, 1)$ exits. Now consider
  $R(s,t)$. Let $G$ be a graph on $R(s-1, t) + R(s, t-1)$
  vertices. Pick a vertex $v \in G$. If $|N(v)| \geq R(s - 1, t)$,
  then either $N(v) + v$ contains a $\bar{K^t}$ or a
  $K^{s}$. Otherwise the non-edges have size at least $R(s, t - 1)$,
  once again implying the existence of $\bar{K^t}$ or $K^s$ in
  $G - N(v)$.
\end{proof}

\begin{theorem} (7.1.1 Turan's Theorem) For all integers $r, n$ with
  $r \geq 1$, every graph $G$ not containing $K^r$ with $n$ vertices
  and $ex(n, K^r)$ edges is $T^{r-1}(n)$.
\end{theorem}
\begin{proof} In this proof I will show that if a graph with the
  maximum number of edges must be a complete partite graph. Since
  $T^{r-1}(n)$ contains the most edges among complete partite graphs
  $K^r$ free graphs, the theorem will have been proved. Let $G$ be a
  graph with the maximum number of edges such that $G$ does not
  contain a $K^r$. Assume that $G$ is not a complete $k$-partite
  graph. Then there exists vertices $x$, $y_1$, and $y_2$ such that
  $xy_1$ and $xy_2$ are not edges in $G$, but $y_1y_2$ is an edge in
  $G$. If $d(x) < y_i$, then deleting $x$ and duplicating $y_i$
  results in a new graph $G'$ which still must be $K^r$ free and has
  more edges than $G$. Otherwise $d(x) \geq y_i$. But then deleting
  $y_1$ and $y_2$ and replacing them with $2$ copies of $x$ once again
  creates a new graph $G'$ which is $K^r$ free with more
  vertices. Thus $G$ must be complete $r$ partite, finishing the
  proof.
\end{proof}

\begin{theorem} (7.1.2 Erdos Stone) For all integers $r \geq 2$ and
  $s \geq 1$, and every $\gamma > 0$, there exists an integer $n_0$
  such that every graph with $n \geq n_0$ vertices and at least
  $t_{r-1}(n) + \gamma n^2$ edges contains $K^{r}_{s}$ as a subgraph.
\end{theorem}
\begin{proof} Let $n_0$ be large enough. By regularity, let
  $\epsilon > 0$, then there is an $\epsilon$-regular partition of $G$
  into sets $V_0, V_1, \cdots V_k$ with $m \leq k \leq M$. Let
  $\ell = |V_i|$ for $i > 0$. Let $d > 0$ and $R$ be the regularity
  graph associated with $d$ (the epsilon regular pairs of $G$ with
  density at least $d$ for edges in $R$). It is known that if $s$ is a
  fixed constant, then $G$ contains $R_s$ when $n_0$ large
  enough. Thus it suffices to show that $R$ must contain $K^r$. Assume
  to the contrary, then $||R|| < t_{r -1}(k)$.For each edge in $R$ we
  can get at most $\ell^2$ edges in $G$. Since we have an epsilon
  regular partition, there fewer than $\epsilon n^2$ edges from
  derived from pairs $V_i, V_j$ which are not epsilon
  regular. Finally, in the non-edges in $R$, there are at most $d n^2$
  possible edges. Thus the number of edges in $G$ is strictly smaller
  than
  $t_{r-1}(k) \ell^2 + (d + \epsilon)n^2 = t_{r-1}(kl) + (d +
  \epsilon)n^2 \leq t_{r-1}(n) + (d + \epsilon)n^2$.  Picking
  $d + \epsilon < \gamma$ yields that
  $||G|| < t_{r - 1}(n) + \gamma n^2$, a contradiction.
\end{proof}

\begin{theorem} (7.2.1) There is a constant $c \in \mathbb{R}$ such
  that for every $r \in \mathbb{N}$, every graph $G$ of average degree
  $d(G) \geq cr^2$ contains $K^r$ as a topological minor.
\end{theorem}
\begin{proof} We will prove this with $c = 10$. From Mader's Theorem
  (1.4.3), we get that $G$ contains an $r^2$ connected graph $H$. Pick
  $r$ distinct vertices $V$ in $G$ as branch vertices along with $r-1$
  neighbors for each branch vertex chosen. Note they must exists since
  $\delta(H) \geq \kappa(H) \geq r^2$. From another theorem, note that
  $H$ is $\frac{1}{2}r^2$ linked (by an unlisted theorem). Thus we can
  pair off the neighbors of the $r$ branch vertices with one from each
  branch and by the linkedness we obtain a $K^r$.
\end{proof}

\begin{theorem} (Blowup Lemma 7.5.2) For all $d \in (0, 1]$ and
  $\Delta \geq 1$, there exists an $\epsilon_0 > 0$ with the following
  property: if $G$ is any graph, $H$ is a graph with
  $\Delta(H) \leq \Delta$, $s \in \mathbb{N}$ and $R$ is any
  regularity graph of $G$ with parameters $\epsilon \leq \epsilon_0$,
  $\ell \geq 2s/d^\Delta$, and $d$., then
  $H \subseteq R_s \implies H \subseteq G$.
\end{theorem}
\begin{proof}
\end{proof}

\begin{theorem} (7.4.1 Regularity) For every $\epsilon > 0$ and every
  integer $m \geq 1$ there exists an integer $M$ such that every graph
  of order at least $m$ admits an $\epsilon$-regular partition
  $\{V_0, V_1, \ldots, V_k\}$ with $m \leq k \leq M$.
\end{theorem}
\begin{proof}
\end{proof}

\begin{theorem} (9.2.2) For every positive integer $\Delta$, there
  exists a constant $c$ such that $R(H) \leq c |H|$, for all graphs
  $H$ with $\Delta(H) \leq \Delta$.
\end{theorem}
\begin{proof} % Let $H$ be a graph with $\Delta(H) = \Delta$. Let $G$
  be a graph with $|G| \geq c |H|$. Let $\epsilon_0$ and $n_o$ be the
  results returned by the blow up level which allow a regularity graph
  to be blown up to a size of $|H|$. For an epsilon regular partition
  of $G$ into $k$ sets with size $\ell$, let $R$ be the regularity
  graph on density $0$. Then
  $||R|| \geq {k \choose 2} - \epsilon k^2$. By choosing $\epsilon$
  small enough, we are guaranteed that $||R|| \geq t_{m - 1}(k)$. Let
  $R'$ be the
\end{proof}

\begin{theorem} (5.4.1) There is a function
  $f:\mathbb{N} \to \mathbb{N}$ such that given any integer $k$, all
  graphs $G$ with average degree $d(G) \geq f(k)$ satisfy
  $ch(G) \geq k$.
\end{theorem}
\begin{proof} Let $G$ be a graph with average degree at least
  $d \geq f(k)$. Then $G$ contains a bipartite subgraph $H$ with
  $\delta(H) \geq \frac{d}{4}$. Such a graph can be obtained from $G$
  by iteratively deleting vertices from $G$ with degree less than
  $\frac{d}{2}$, until no other vertices remain and letting $H$ be the
  bipartite graph with the maximum number of edges. Note that by the
  definition of $H$, it is the induced bipartite graph, and if there
  is a vertex with degree less than $\frac{d}{4}$, moving that vertex
  to the other partition will induce a bipartite graph with more
  edges. Let $(A,B)$ be the bipartition of $H$ and assume
  $|A| \geq |B|$. Let $S$ be a set of $s^4$ colors. We will show that
  there is a $s$-list assignment of $H$ with colors taken from $S$
  such that $H$ contains no proper coloring. This immediately implies
  that $ch(G) \geq ch(H) > s$, completing the proof.

  Consider a a random list assignment, where for each vertex
  $b \in B$, $S(b)$ is a random list of size $s$ chosen independently
  and uniformly from $S$. Call a vertex in $A$ good if for every
  subset $C \subset S$ with $|C|=S$, there exists a $b \in N(a)$ with
  $S(b) = C$. The probability that a vertex $a$ is not good is
  \[
    {s^4 \choose s} {(1 - \dfrac{1}{{s^4 \choose s}})}^{\frac{d}{4}}
    \leq \dfrac{1}{2}
  \]
  by choice of $d$.

  Fix a list assignment of $B$ such that there are at least $|A|/2$
  good vertices, which exists since the probability a list is good is
  at least $1/2$. Now consider a random $s$-list assignment for each
  $a \in A$. We will show that with positive probability there is not
  proper coloring of $H$.

  There are $s^{|B|}$ possible colorings of $B$. For a fixed coloring,
  consider the probability that this coloring can be extended to the
  vertices in $A$. Consider a good vertex $a \in A$. Since every
  possible $s$ subset of $S$ is in the neighborhood of $a$, at most
  $s-1$ colors can be assigned to $a$ given the coloring $c$
  restricted to $B$, as otherwise some subset of ${S \choose s}$ has
  no elements in $c$ restricted to $B$, contradicting the goodness of
  $a$. Thus a proper coloring of $H$ can only exists if $S(a)$
  contains one of the remaining colors $s-1$ colors. The probability
  that a randomly chosen subset satisfies this is at most
  \[
    \dfrac{(s-1) {s^4 -1 \choose s-1}}{{s^4 \choose s}} =
    \dfrac{s(s-1)}{s^4} < \dfrac{1}{s^2}
  \]
  Since there are at least $|A|/2$ good vertices, the chances that
  there is a proper coloring is strictly less than
  \[
    {(\frac{1}{s^2})}^{|A|/2} \leq \dfrac{1}{s^{|B|}}
  \]
  Since there are only $s^{|B|}$ colorings of $B$, with positive
  probability, there exists a list assignment of $A$ such that $A$ is
  not colorable for any coloring of $B$.
\end{proof}

\begin{theorem} (11.1.3) For every integer $k \geq 3$, the Ramsey
  number of $k$ satisfies $R(k) \geq 2^{k/2}$.
\end{theorem}
\begin{proof} For $k = 3$, it is true that
  $R(3) \geq 3 > 2^\frac{3}{2}$. So assume that $k \geq 4$. Consider a
  random edge-coloring of $K^{n}$ for $n \leq 2^{\frac{k}{2}}$ by two
  colors, where each color is chosen with probability $\frac{1}{2}$
  for every edge. Let $r$ denote the size of the largest clique of
  color $1$ and $s$ denote the size of the largest clique of number
  $2$. Then
  \[
    \begin{array}{rcl} \Pr(r \geq k) & = & \Pr(s \geq k) \leq {n
                                           \choose k} 2^{-{k \choose 2}}\\
                                     & < & \dfrac{n^k}{2^k}
                                           2^{\frac{-k(k-1)}{2}}\\
                                     & = &
                                           \dfrac{2^{\frac{k^2}{2}}}{2^k}
                                           2^{\frac{-k(k-1)}{2}}\\
                                     & = & 2^{\frac{-(k-1)}{2}}\\
                                     & < & \dfrac{1}{2}
    \end{array}
  \]
  Since $\Pr(r \geq k) + \Pr(s \geq k) < 1$, we get that there must
  exist a coloring of $K^{n}$ such that no clique of size $k$ exists
  for either color implying $R(k) \geq 2^{\frac{k}{2}}$.
\end{proof}

\begin{theorem} (11.2.2) For every integer $k$, there exists a graph
  $H$ with girth $g(H) > k$ and chromatic number $\chi(H) > k$.
\end{theorem}
\begin{proof} Assume that $k \geq 3$ and let
  $0 < \epsilon < \frac{1}{k}$. Let $p = n^{\epsilon - 1}$. Let $X$ be
  the expected number of cycles with size at most $k$. Then we get
  that
  \[
    E[X] = \sum_{i = 3}^{k} \dfrac{{(n)}_i}{2i} p^i \leq \dfrac{1}{2}
    \sum_{i = 3}^{k} n^i p^i \leq \dfrac{1}{2} (k - 2) n^k p^k
  \]


  Note the last inequality holds since $np = n^\epsilon > 1$. But then
  from Markov's inequality we get that
  \[
    \Pr(X \geq \frac{n}{2}) \leq \dfrac{E[X]}{\frac{n}{2}} \leq
    (k-2)n^{k - 1} p^k = (k-2) n^{k \epsilon - 1}
  \]


  By choice of $\epsilon$, $k \epsilon - 1 < 0$. Thus we can pick $n$
  such that $\Pr(X \geq \frac{n}{2}) \leq \frac{1}{2}$.

  Now consider $\Pr(\alpha \geq \frac{n}{k})$.
  \[
    \Pr(\alpha \geq \frac{n}{2k}) \leq {n \choose \frac{n}{2k}}
    p^{\frac{n}{2k} \choose 2} \leq n^{\frac{n}{2k}}
    n^{\frac{{(\frac{n}{2k} - 1)}^2}{2}(\epsilon - 1)}
  \]
  From this we can see that when $n$ is large enough
  $\Pr(\alpha \geq \frac{n}{2k}) < \frac{1}{2}$. Choose $n$ large
  enough such that $\Pr(X \geq \frac{n}{2} < \frac{1}{2})$ and
  $\Pr(\alpha \geq \frac{n}{2k} < \frac{1}{2})$. Then there exists a
  graph $G$ on $n$ vertices with $X < \frac{n}{2}$ and
  $\alpha < \frac{n}{2k}$. For each cycle counted in $X$, delete a
  vertex to create a graph $G'$. But then
  \[
    \chi(G') \geq \dfrac{|G'|}{\alpha(G')} >
    \dfrac{\frac{n}{2}}{\frac{n}{2k}} = k
  \]

\end{proof}

\section{Example Problems}
\begin{problem} For any integer $n \geq 3$ and onto coloring
  $f: E(K^n) \to [n]$ of the edges of $K^n$, show that $K^n$ contains
  a rainbow triangle, a $K^3$ with all edges a distinct color.
\end{problem}
\begin{proof} Let $C$ be a minimum size rainbow cycle in $K^3$ under
  the coloring $f$. Note such a cycle must exists because there are at
  least $n$ edges with distinct colors, implying a subgraph of $K^n$
  with one edge of each color must contain a a rainbow cycle. If $C$
  is a $K^3$, then we are done. Otherwise $C$ has a chord $xy$ with
  some color $a$. The edge $xy$ is contained in two cycles $C_1$ and
  $C_2$ using the edges in $C$ and $xy$ itself. Since $C$ is a rainbow
  triangle, one of the cycles $C_i$ must not contain the color $a$ on
  any of its edges in $C_i \cap C$. But then $C_i$ is a rainbow cycle,
  contradicting the minimality of $C$.
\end{proof}

\begin{problem} Let $k$ be a positive integer. Let $G$ be a graph with
  $\epsilon(G) \geq k$.
  \begin{itemize}
  \item[(a)] Let $G'$ be a minimal minor of $G$ with
    $\epsilon(G') \geq k$. Show that for every vertex $v$ in $G'$ and
    every vertex $u \in N_{G'}(v)$,
    $|N_{G'}(u) \cap N_{G'}(u)| \geq k$.
  \item[(b)] Show that $G$ contains a minor $H$ such that
    $\delta(H) \geq \min(k, \frac{|H|}{2})$.
  \end{itemize}
\end{problem}
\begin{proof}
  \begin{itemize}
  \item[(a)] Let $G'$ be a minimal minor of $G$ with
    $\epsilon(G') \geq k$. Pick an edge $uv$ in $G'$ and assume for
    contradiction that $|N_{G'}(u) \cap N_{G'}(u)| < k$. Consider the
    graph $G'/uv$. Note that $G'$ contains exactly one fewer vertices,
    and at most $k$ fewer edges. But then $e(G'/uv) \geq k$,
    contradicting the minimality of $G'$. Thus we get that
    $|N_{G'}(u) \cap N_{G'}(u)| \geq k$.
  \item[(b)] Let $G'$ be a minimal minor of $G$ with
    $\epsilon(G') \geq k$. Since $G$ is a minor of itself with
    $\epsilon(G) \geq k$, such a $G'$ must exist. From $a$ we get that
    $\delta(G') \geq k$.
  \end{itemize}
\end{proof}

\begin{problem} Show that for every positive integer $k$ there exists
  an integer $n$ such that for every tournament $T$ on $n$ vertices
  and every coloring $f: E(T) \to N$, there exists a subtournament
  $S \subseteq T$ on $k$ vertices such that either for every vertex
  $v \in V(S)$ all the out edges of $S$ incident to $v$ have the same
  color or for every vertex $v \in V(S)$ all the out edges of $S$
  incident of $v$ have different colors.
\end{problem}
\begin{proof} Let $R(s,t)$ be the minimum value $n$ such that any
  tournament $T$ on $n$ vertices and every coloring $f:E(T) \to N$
  contains either a subtournament $S$ on $s$ vertices such that for
  all $v \in S$, all out edges of $S$ incident to $v$ have the same
  color, or $T$ contains a subtournament $S$ on $t$ vertices such that
  for all $v \in V$, all out edges of $S$ incident to $v$ have a
  different color.

  Note that the claim holds for $R(1,t) = R(s, 1) = 1$. Fix $s,t$ with
  $s,t > 1$. Let $T$ be a tournament on at least $2R(s-1)R(s,t-1) + 1$
  vertices and $f$ a coloring of the edges of $T$. Let $v$ be a vertex
  in $T$ and $N^{-}_c(v)$ denote set vertices whose out edges incident
  to $v$ have color $c$ under $f$. If there exists a color $c$ such
  that $N^{-}_c(v) \geq R(s - 1, t)$, then by the inductive hypothesis
  we find a tournament $S$ in $N^{-}_c(v)$ such that either $S + v$ or
  $S$ satisfies the claim for $T$. Otherwise for all vertices
  $v \in V(T)$ and all colors $c$, we get that
  $|N^{-}_c(v)| < R(s-1, t)$. Let $v$ be a vertex in $T$ such that
  $|N^{-}(v)| \geq |T|/2$ and note that such a vertex must exists. Let
  $A$ be a maximum sized set of vertices such that for all $a$ in $A$,
  $av$ is an edge in $T$ and for all $a, a'$ in $A$,
  $f(av) \neq f(a'v)$. Note that if $|S| \geq R(s, t-1)$, then we are
  once again done by induction. Assume that $|S| < R(s, t-1)$. Since
  $|N^{-}_c(v)| < R(s-1, t)$, every vertex $s$ in $S$ corresponds to
  at most $ R(s-1, t)$ other vertices with out edges $e$ incident to
  $v$ with $f(e) = f(sv)$. Thus
  \[
    |G| < |S| R(s, t-1) + |N^{+}(v)| + 1 \leq R(s-1, t)R(s, t-1) +
    |G|/2 + 1
  \]
  But then $|G| \leq 2R(s-1)R(s,t-1) < |G|$, a contradiction, thus
  completing the proof.

\end{proof}

\begin{problem} Let $a \in N$ and let $G=(V, E)$ be a graph with
  arboricity $a$. Show that if $B = \{ v \in V | d(g) \geq 5a\}$ and
  $C$ is a vertex cover of $G$, then $|C| \geq 4|B|/5$.
\end{problem}
\begin{proof} Assume to the contrary and let $X = B - C$. Then we
  immediately know that $|X| > |B|/5$ and that $X$ is an independent
  set. In particular we get that
  \[
    ||G[X \cup N(X)]|| \geq 5a |X|
  \]
  By Nash-Williams theorem, we know that
  \[
    ||G[X \cup N(X)]|| \leq a (|X \cup N(X)| - 1)
  \]
  combining these inequalities gives $4|X| < |N(X)|$ so
  \[
    \frac{4}{5} |B| < 4 |X| < |N(X)| \leq |C|
  \]
  contradicting our assumption.
\end{proof}

\begin{problem} Let $G = (V, E)$ be a planar graph.
  \begin{itemize}
  \item[(a)] Let $e \in E$. If there are no separating triangles in
    $G$, then how many triangles can contain $e$.
  \item[(b)] Show that there is a $5$-coloring of $E$ such that every
    triangle is a rainbow triangle.
  \end{itemize}
\end{problem}
\begin{proof}
\item[(a)] Consider an edge $e=xy$ contained in $3$ triangles. Let
  $v_1$, $v_2$ and $v_3$, be the vertices of these triangles not
  contained in $e$. WLOG, Note that there cannot be a path from $v_1$
  to $v_2$ without using $x$, $y$, or $v_3$ as this will generate a
  $TK^5$ which is not planar. Thus $v_1 xy$ is a separating triangle,
  implying that an edge $e$ is contained in at most two triangles in a
  graph $G$ with no separating triangles.
\item[(b)] The proof will by by induction on $|G|$ and $||G||$. Note
  that if $G$ is the empty graph, then we are done. Otherwise, assume
  that $G$ contains no separating triangles. Let $e$ be an edge in
  $G$. Then $G-e$ can be colored such that all triangles are rainbow
  colored. Since $e$ is contained in at most $2$ triangles, there are
  at most $4$ colors used on the edges of the triangles, as such we
  can color $e$ with the remaining color to construct a coloring of
  $G$ where every triangle is rainbow colored. Otherwise, let $T$ be a
  separating triangle. Let $C_i$ be the $i$th component of $G-T$ and
  let $D_i = G[V(C_i) \cup T]$. By induction each $D_i$ can be colored
  such that every triangle is rainbow colored. By permuting the
  colorings of $D_i$, we can obtain a coloring such that all the
  colorings are the same on $T$ in each $D_i$. Combining these
  colorings to create an edge coloring on $G$ provides the desired
  coloring.
\end{proof}

\begin{problem} Let $G$ be an $A,B$-bigraph with minimum degree
  $\delta \geq 1$. Show that $G$ has a spanning subgraph $H$ such that
  for every vertex $a \in A$, $d_H(a) = 1$ and for every vertex
  $b \in B$, $d_H(b) \leq \lceil d_G(b) / \delta \rceil$.
\end{problem}
\begin{proof} Consider a new bipartite graph $G'$ with partition
  $A-B'$ obtained from $G$ by splitting the neighbors of each vertex
  $b \in B$ into $\lceil d_G(b) / \delta \rceil$ parts $P_i$ of size
  at most $\delta$, creating a new vertex $b_i \in B'$ in $G'$ and
  adding the edges edges from $b_i$ to vertices in $P_i$.  Note that
  if we can find a matching of $A$ onto $B'$, this will correspond to
  the desired subgraph of $G$. Consider any set $S \subseteq A$. Then
  we know that the number of edges leaving $S$ is at least
  $\delta |S|$. Since $\delta$ is the maximum degree of any vertex in
  $B$ though we get that $|S| \leq |N(S)$. Therefore by Hall's
  theorem, there is a perfect matching of $A$ onto $B'$.
\end{proof}

\begin{problem} Prove that for all $t,w \in Z^+$, there exists an
  $k \in \mathbb{N}$ such that for every graph $G$ that does not
  induce $K_{1,t}$ and satisfies $\omega(G) \leq w$.
  \begin{itemize}
  \item[(a)] $\chi(G) \leq k$
  \item[(b)] if $w = t = 3$, then $\chi(G) \leq 5$.
  \end{itemize}
\end{problem}
\begin{proof}
\end{proof}

\begin{problem} Suppose that $G$ has a $4$-flow. Prove that there
  exists a multiset $\mathcal{C}$ consisting of cycles in $G$ such
  that every edge is in exactly two cycles in $\mathcal{C}$.
\end{problem}
\begin{proof} The proof will by by induction on $||G||$. To begin with
  note that if $||G|| = 0$, the claim holds. Now assume $||G|| >
  0$. To begin with, since $G$ has a $4$-flow, it must contain a flow
  on the Klein group $\mathbb{Z}_2\times \mathbb{Z}_2$. Let $G_i$ be
  the graph induced by the edges with a non-zero flow on coordinate
  $i$. Then each $G_i$ has a $2$-flow and must be an even graph. Since
  $G$ is nonempty, we may assume that $||G_1|| > 0$. Thus there exists
  a cycle $C$ in $G_1$. If $C$, a cycle in $G_1$, is edge disjoint
  from any cycle in $G_2$, then there exists a $4$-flow in $G -
  E(C)$. Thus by the inductive hypothesis there exists family
  $\mathcal{C}'$ containing every edge in $G-E(C)$ twice. Adding $C$
  to $\mathcal{C'}$ twice creates the desired set
  $\mathcal{C}$. Otherwise there exists a cycle $C'$ in $G_2$ with
  containing a strict subset of the edges of $C$. Note that
  $G - E(C) - E(C)$ once again has a $4$-flow so from the inductive
  hypothesis so it suffices to find cycles to add to the set
  $\mathcal{C}'$. Let $G''$ be the set of edges in exactly one of $C$
  or $C'$. Note that $G''$ must be an even graph. Thus we can
  decompose $G''$ into cycles. Adding $C$, $C'$, and $G''$
  decomposition to $\mathcal{C}'$ then generates the desire set
  $\mathcal{C}$.
\end{proof}

\begin{problem} Show that for all positive integers $k$ there exists
  an integer $t$ such that if $G$ is a graph with a Hamiltonian path,
  then either $G$ contains $k$ edge-disjoint cycles or there exists a
  $t$-set $F \subseteq E(G)$ such that $G-F$ is acyclic.
\end{problem}
\begin{proof} The proof will be by induction on $||G||$. To begin
  with, note that if $G$ is a path, the claim is true. For
  contradiction, assume that $G$ does not contain $k$ edge-disjoint
  cycles. To begin, let $e$ be an edge in $G$ which is not contained
  in any $K_3$. Note that $G/e$ must not contain $k$ edge-disjoint
  cycles and by induction there is a set of at most $t$ edges $T$ such
  that $G/e - T$ is acyclic. But then $G-T$ must be acyclic as well
  since no cycles were destroyed by contracting $e$. Otherwise, every
  edge $e$ belongs to some $K3$. In particular, consider edges on the
  Hamiltonian path $P$. Note that every edge $e$ corresponds to some
  $K_e$ and there are at most two edges $e$ and $e'$ such that
  $K_e = K_{e'}$. Starting from one end of $P$, construct a set $K$ of
  $K_3$ which contains $K_e$ for every other edge $e \in P$. Note that
  each of these $K_e$ must be edge-disjoint all other edges in $K_e$
  contain a vertex incident to $e$, which is in no other considered
  vertex. But then $|G| = |P| < 2k$, and $G$ can be made a cyclic by
  deleting at most ${2k \choose 2}$ edges.
\end{proof}
\begin{problem} Show that if $G$ is a graph with order at least $4$
  and $G$ contains at most one vertex with degree at most two, then
  $G$ contains a $TK^4$.
\end{problem}
\begin{proof} The proof will by by induction on $|G|$. For the base
  case of $|G|=4$, note that the only graph satisfying the assumption
  is $K^4$ so we are immediately done.

  Now let $G$ be a graph of order at least $5$ with at most one vertex
  of degree at most $2$. If $G$ contains a vertex $v$ of degree at
  most $1$, then $G-v$ satisfies the inductive hypothesis thus
  containing a copy of $TK^4$ which is contained in $G$. Thus $G$
  contains exactly one vertex of degree at least $2$. Assume that $G$
  has no vertex of degree $2$, then $G/e$ for any edge $e$ satisfies
  the inductive hypothesis, thus $G/e$ contains a $TK^4$ corresponding
  to a $TK^4$ in $G$ and we are done. Finally, $G$ contains exactly
  one vertex $v$ of degree 2 to with neighbors $xy$. Note that $G/xv$
  satisfies the inductive hypothesis and thus completes the proof
  unless $xy \in E(G)$ and $d(x) = d(y) = 3$. If $N(x) = N(y)$, then
  $G - x - x - y - v$ contains exactly one vertex $v'$ of degree at
  most $3$. Also we get that $v'$ is connected to at least one other
  vertex $u$ which has degree at least $3$ implying there are at least
  $4$ remaining vertices so $G - x - y - v$ contains a $TK^4$ by the
  inductive hypothesis, implying $G$ does as well. As a final case,
  consider when $N(x) \neq N(y)$. Then $(G - v)/xy$ satisfies the
  inductive hypothesis, completing the proof.
\end{proof}

\begin{problem} Two distinct vertices $u,v$ are called twins if
  $N(u) = N(v)$ and a graph is called twinless if it has no
  twins. Show that a connected twinless planar graph $G$ on at least
  two vertices has a matching $M$ with $|M| \geq |G|/8$.
\end{problem}
\begin{proof}
\end{proof}

\begin{problem}
  \begin{itemize}
  \item[(a)] Show that if a graph $G$ has $\delta(G) \geq 2$, then $G$
    has a cycle $C$ such that at least one vertex in $C$ has no
    neighbors in $G-C$.
  \item[(b)] Show that vertices of the graph $G$ can be covered by at
    most $\alpha(G)$ pairwise disjoint graphs each isomorphic to a
    cycle, $K_2$ or $K_1$.
  \end{itemize}
\end{problem}
\begin{proof}
\item[(a)] Let $P = x_0\cdots x_k$ be a maximum length path in
  $G$. Let $i$ be the smallest integer such that $x_i x_k$ is an
  edge. Since $P$ is maximal, $N(x_k) \subseteq V(P)$, because
  otherwise we could extend the path. But then $x_i P x_k$ forms a
  cycle $C$ with $N(x_k) \subseteq V(C)$.
\item[(b)] Construct a sequence of graphs $G_0$ = $G$ and $G_k$ is the
  empty graph. Construct $G_{i + 1}$ from $G_i$ as follows. Let $P$ be
  a maximum length path with $x_i$ the endpoint of maximum degree. If
  $d(x_i) \geq 2$, then there is a cycle $C$ such that
  $N(x_i) \subset C$ by the same argument as in $(a)$. Let
  $G_{i+1} = G - C$ and not $N(x_i) \cap G_{i + 1} =
  \emptyset$. Otherwise, if $d(x_i) = 1$, let $y$ denote the neighbor
  of $x_i$ and set $G_{i+1} = G - x_i - y$ and note once again that
  $N(x_i) \cap G_{i+1} = \emptyset$. Finally if $d(x_i) = 0$ let
  $G_{i + 1} = G - x_i$. Note that at every step, $x_i$ is independent
  form all previously found $x_j$, so the set of $x_i$ forms an
  independent set. Thus the procedure was repeated at most $\alpha(G)$
  times, providing the desired vertex cover.
\end{proof}

\begin{problem} Prove that every graph $G$ is even or has a vertex
  partition ${X,Y}$ such that $G[X]$ and $G[Y]$ are both even. (Hint:
  Consider a vertex $v$ with $d(v)$ odd and modify $G-v$ by changing
  $G[N(v)]$ to its complement).
\end{problem}
\begin{proof} The proof will be by induction on $|G|$. To begin with,
  note that if $G$ is even, then we are done. Otherwise, let $x$ be a
  vertex in $G$ with odd degree. Construct a graph $G'$ from $G - x$
  by replacing $G[N(x)]$ with its complement. Then $G'$ contains a
  partition $X,Y$ such that $G'[X]$ and $G[Y]$ are even, although $Y$
  could be the empty set. WLOG assume that $|N(x) \cap X|$ is
  odd. Then $G[X]$ is still even. On the other hand $G[Y]$ is odd for
  only the vertices in $N(x)$. But since there are an even number of
  neighbors of $x$ in $Y$, then $X$ and $Y+x$ gives the desired
  partition of $G$.
\end{proof}
\begin{problem} Let $G$ be a balanced $X,Y$-bigraph such that $(X,Y)$
  is an $\epsilon$-regular pair. Set $n= |X|=|Y|$ and $d=(X,Y)$. Show
  that if $0 < \epsilon < d < 1$, then $G$ has a vertex partition
  $\{V_o, V_1, \cdots V_k\}$ such that
  $|V_0| \leq 2 \max(\frac{\epsilon n}{d - \epsilon},
  2{(d-\epsilon)}^{-2})$ and $G[V_i] = C_4$ if $i \in [k]$.
\end{problem}
\begin{proof} Let $\{ V_0, V_1, \cdots V_k\}$ with $G[V_i] = C_4$ for
  $i \in [k]$ with $k$ maximized. Assume for contradiction that
  $|V_0| > 2 \max(\frac{\epsilon n}{d - \epsilon},
  2{(d-\epsilon)}^{-2})$. Let $X_0 = V_0 \cap X$ and
  $Y_0 = Y \cap V_0$. Note that $|X_0| = |Y_0|$ as each $V_i$ must
  contain an equal number of vertices from each of $X$ and $Y$. From
  the assumption, we know that
  $|X_0| = |Y_0| > \frac{\epsilon n}{d - \epsilon}$, and thus that
  $|d(X_0, Y_0) - d(X, Y)| \leq \epsilon$, and thus we know that
  $d_0 = d(X_0, Y_0) \geq d - \epsilon$.

  Now note that the number of pairs of vertices in $Y_0$ is bounded by
  \[
    {|Y_0| \choose 2 } \leq \dfrac{|Y_0|^2}{2}
  \]
  On the other hand, consider the number of such pairs contained in
  $N(x)$ of $x$ over all such vertices $x$, then
  \[
    \sum_{x \in X} {|N(x)| \choose 2} \geq \sum_{x \in X} {\lfloor
      E[|N(x)|] \rfloor \choose 2 } \geq |X_0| {\lfloor (d - \epsilon)
      |Y_0| \rfloor \choose 2}
  \]
  Since there are no copies of $C_4$ contained in $G[X_0, Y_0]$, we
  know that
  \[
    |X_0| {\lfloor (d - \epsilon) |Y_0| \rfloor \choose 2}\leq
    \dfrac{|Y_0|^2}{2}
  \]
  If
  $(d - \epsilon) |Y_0| 2^{-1/2} < \lfloor (d - \epsilon) |Y_0|
  \rfloor$, from these two inequalities we get that
  \[
    \dfrac{{(d - \epsilon)}^2 |Y_0|^3}{4} \leq \dfrac{|Y_0|^2}{2}
  \]
  implying $|Y_0| \leq 2{(d - \epsilon)}^{-2}$, a contradiction.  Thus
  $(d - \epsilon) |Y_0| 2^{-1/2} \geq \lfloor (d - \epsilon) |Y_0|
  \rfloor$, but then $(d - \epsilon) |Y_0| < 3$. On the other hand,
  since $d-\epsilon < 1$, we get that $|Y_0| \geq 2$. But then going
  back to the original counts, we get that
  ${2 \choose 2} \geq 2 {2 \choose 2}$, a contradiction again.
\end{proof}

\begin{problem} Let $G=K_{2t - 1} - M$ where
  $V(G) = \{ x_1, \cdots x_t\} \cup \{y_2, \cdots y_t\}$ and $M$ is
  the set of edge $x_i y_i$. Suppose that $L$ is a list assignment for
  $G$ such that $|L(x_i)| = t$ for all $i \in [t]$ and
  $|L(y_i)| = t-1$ for all $i \in [t]\backslash \{1\}$. Show that $G$
  has an $L$-coloring.
\end{problem}
\begin{proof} The proof will be by induction on $t$. To begin with,
  note that the claim trivially holds for $t= 1$. Otherwise, assume
  there exists a pair $x_i$ and $y_i$ and a color $\alpha$ such that
  $\alpha \in L(x_i) \cap L(y_i)$. Remove the color $\alpha$ from all
  list in $L$ to create a new list assignment $L'$. But then $L'$ on
  $G-x_i -y_i$ satisfies the inductive hypothesis, we can color
  $G-x_i-y_i$ and then color $x_i$ and $y_i$ with the color $\alpha$
  to obtain a proper coloring of $G$. Otherwise, for all pairs
  $x_i, y_i$, $L(x_i) \cap L(y_i) = \emptyset$. Construct a bipartite
  graph from the vertices of $G$ to colors listed in $L$, were the
  edge $v c$ exists iff $c \in L(v)$. The current goal is to find a
  matching from $V(G)$ to colors $c$. Let $S$ be a subset of
  $V(G)$. If $|S| < t-1$, then by definition of $G$,
  $N(S) \geq t-1 \geq |S|$. If the $|S| = t$, then there must exists a
  vertex $x_i \in S$ by pigeon hole principle, thus
  $N(S) \geq t = |S|$. Finally let $S > t$. Then by pigeon hole
  principle there must exists a pair $x_i$ $y_i$ with
  $x_i, y_i \in S$. But then $N(S) \geq 2t - 1 \geq |S|$. Thus there
  is a matching from $V(G)$ to colors, providing a proper coloring of
  $G$.
\end{proof}

\begin{problem}
  \begin{itemize}
  \item[(a)] Suppose that $G'$ is a $k$-connected subgraph of $G$ and
    $v \in V(G) \backslash V(G')$ is such that
    $|N(v) \cap V(G')| \geq k$. Show that $G[V(G') \cup \{ v\}]$ is
    $k$-connected.
  \item[(b)] Suppose $G$ is a graph with
    $\delta(G) \geq \frac{1}{2}|G|$. Show that there exists a
    partition $\mathcal{P}$ of $G$ into at most $2$ parts such that
    $G[X]$ is $\frac{|G|}{100}$ connected for each
    $X \in \mathcal{P}$.
  \end{itemize}
\end{problem}
\begin{proof}
  \begin{itemize}
  \item[(a)] Consider a separating set $X$ in
    $H = G[V(G') \cup \{ v \}]$ and assume for contradiction that
    $|X| < k$. Let $C$ be the component in $H - X$ containing
    $v$. Then $V(C) = x$, as otherwise $X$ would be a separating set
    in $G'$ as well, contradicting that $G'$ is $k$-connected. But
    then $N(v) \subseteq X$, once again implying that $|X| \geq
    k$. Thus $H$ is $k$-connected.
  \item[(b)] To begin, if $G$ is $|G|/100$ connected, then we are
    done. Otherwise let $X$ be a separating set with $|X| <
    |G|/100$. Then $G-X$ is made up of components $C_i$. From the
    minimum degree of $G$ with get that $G[V(C_i) \cup X]$ must
    contain more than $|G|/2$ vertices. Thus there are only two
    components $C_i$ as $|V(C_i)| \geq \frac{49}{100} |G|$. But then
    we also get that $|V(C_i)| \leq \frac{50}{100} |G|$. Let $Y$ be a
    separating set of $C_i$ and let $x_1$ and $x_2$ be two vertices in
    separate components. Note that in $G[V(c_i)]$,
    $\delta(x_i) \geq \frac{49}{100} |G|$. As such,
    $|Y| \geq |N(x_1) \cap N(x_2)| \geq \frac{47}{100} |G|$, thus
    $G[V(C_i)]$ is $|G|/100$ connected. Let $P_1$ be the set of
    vertices in $C_1$ and vertices in $X$ with a majority of their
    neighbors in $P_1$. Let $P_2$ be the rest. From part $(a)$, we get
    that $P_1$ and $P_2$ must be $|G|/100$ connected as for any
    $x \in X$, $|N(x) \cap V(C_i)| \geq \frac{24}{100} |G|$ for some
    $i$.
  \end{itemize}
\end{proof}

\begin{problem} Suppose that $0 < \alpha < 1$ and
  $k \in \mathbb{Z}^+$. Show that there exists an integer $n_0$ such
  that for any graph $G=(V,E)$ with $|G| \geq n_0$ and
  $\delta(G) \geq \alpha|G|$, if $f: E \to [k]$, then $G$ contains a
  $4$-cycle $C$ such that $|f^{-1}(C)| = 1$.
\end{problem}
\begin{proof} Fix $f$ and let $c$ be the color that is assigned most
  often by $f$ in $G$. Let $G'$ be the graph composed of all edges in
  $G$ with color $c$. Then
  $||G'|| \geq \frac{||G||}{k} \geq \frac{\alpha |G|^2}{k}$.  Note
  that if there exists two vertices $v_1$ and $v_2$ and two more
  vertices $u_1$ and $u_2$ such that
  $u_i \in N_{G'}(v_1) \cap N_{G'}(v_2)$, then we are done as
  $v_1u_1v_2u_2$ form a monochromatic $C_4$. Then
  \[
    {|G| \choose 2} \geq \sum_{v \in V(G)} {d(v) \choose 2} \geq
    \sum_{v \in V(G)} {\lfloor \frac{\alpha}{k} |G| \rfloor \choose 2}
    = |G|{\lfloor \frac{\alpha}{k} |G| \rfloor \choose 2} \geq
    \frac{\alpha}{2k} |G| {|G| \choose 2}
  \]
  But since $\alpha$ and $k$ are fixed, $|G| \leq \frac{2k}{\alpha}$
  for the above inequality to be true. Thus if
  $|G| \geq \frac{2k}{\alpha}$, there must exists such vertices
  $v_1, v_2, u_1, u_2$ giving a monochromatic $C_4$.
\end{proof}

\begin{problem} Let $G$ be a planar graph with $\delta(G) \geq
  3$. Show that $G$ contains an induced matching of size at least
  $|G| / 16$.
\end{problem}
\begin{proof}
\end{proof}

\begin{problem} Show that for every $\epsilon > 0$, there exists a
  $c > 0$ such that for ever graph $G$ with
  $||G|| \geq \epsilon |G|^2$ has at least $c |G|^5$ copies of
  $K_{2,3}$.
\end{problem}
\begin{proof}
\end{proof}

\begin{problem} A $k$-dominating set in a graph $G=(V,E)$ is a set
  $W \subseteq V$ such that every vertex $v \in V$ is within distance
  $k$ of $W$.
  \begin{itemize}
  \item[(a)] Show that for every $c$ there is a $D$ such that if $G$
    is a graph with $\delta(G) \geq c|G|$, then $G$ has a
    $1$-dominating set of size at most $D \log |G|$.
  \item[(b)] Show that for every $c$ there is a $D$ such that if $G$
    is a graph with $\delta(G) \geq c|G|$, then $G$ has a dominating
    set of size at most $D$.
  \end{itemize}
\end{problem}
\begin{proof}
  \begin{itemize}
  \item[(a)] Let $S$ be a random set of $c \ln |G|$ vertices where
    each vertex is chosen with probability $p$. Then $v$ is not
    dominated iff $(\{v\} \cup N(V)) \cap S = \emptyset$. Thus the
    probability that $v$ is not dominated is at most
    \[
      {(1 - c)}^{c \ln |G|} \leq e^{-\ln(n)} \leq \dfrac{1}{n}
    \]
    But then the expected number of non-dominated vertices $1$. Thus
    there exists a set of size $c \ln |G| + 1$ that dominates all of
    $G$, thus proving the theorem.
  \item[(b)] Let $S$ be a random set of size $\frac{1}{c}$. As before
    the probability that a vertex $v$ is not dominated is bound by
    \[
      {(1 - c)}^{\frac{1}{c}} \leq e^{- 1/ c^2} < c
    \]
    But then the expected number of non-dominated vertices is less
    then $c |G|$. Fix $S$ such that that is the case. Since
    $\delta(G) \geq c|G|$, we immediately get that the remaining
    vertices have a neighbor in the 1-dominated vertices, so $S$
    2-dominates all vertices.
  \end{itemize}
\end{proof}

\begin{problem} Let $G=(V,E)$ be a graph with $G: V \to
  \mathbb{N}$. Then $G$ has an orientation such that
  $d^-(v) \geq g(v)$ for all $v \in V$ iff every induced subgraph
  $H \subseteq G$ satisfies
  \[
    ||H|| + ||H, G-H|| \geq \sum_{v \in V(H)} g(v)
  \]
  (consider an orientation minimizing the error).
\end{problem}
\begin{proof}
\end{proof}

\begin{problem} A graph $G=(V,E)$ is called twinless if for every
  $u,v \in V$ with $u \neq v$, then $N(u) \neq N(v)$. Show that for
  every $h$ there is a $c > 0$ such that every twinless graph of order
  $n \geq 2$ with no $TK_h$ has a matching of size at least $cn$.
\end{problem}
\begin{proof}
\end{proof}

\begin{problem} Prove that every graph $G=(V,E)$ has an independent
  set of size $\sum_{v \in V} \frac{1}{1 + d(v)}$.
\end{problem}
\begin{proof} For a given ordering of vertices, we can construct an
  independent set by greedily adding a vertex $v$ to the independent
  set if $v$ has no neighbors preceding itself in the given order. Let
  $X$ be the size of the generated independent set given a random
  ordering of vertices $v$. Then by linearity of expectation we get
  that
  \[
    E[X] = \sum_{v \in V} \Pr(v \text{ precedes its neighbors}) =
    \sum_{v \in V} \frac{1}{1 + d(v)}
  \]
  Thus there must exist some ordering which generates in independent
  set of size at least $\sum_{v \in V} \frac{1}{1 + d(v)}$.
\end{proof}

\begin{problem} Let $(U,V)$ be an $\epsilon$-regular pair such that
  $|U| = |V| \geq \frac{1}{\epsilon}$ and
  $d:= d(U,V) > 2 \sqrt{\epsilon}$. Show that all but at most
  $\epsilon|U|$ vertices in $U$ have at least $(d - \epsilon )|V|$
  neighbors in $V$ and all but at most $3\epsilon |U|^2$ pairs
  $(u, u') \in U \times U$ are such that
  $|N(u) \cap N(u')| \geq (d^2 - 2\epsilon)|V|$.
\end{problem}
\begin{proof}
\end{proof}

\begin{problem} Let $k \geq 2$. Show that in a $k$-connected graph,
  any $k$ vertices lie on a common cycle.
\end{problem}
\begin{proof} Let $G$ be a $k$-connected graph for $k \geq 2$. Let $A$
  be a set of $k$ vertices, and $C$ a cycles in $G$ with
  $|V(C) \cap A|$ maximized. Since $k \geq 2$, $G$ contains a cycle,
  so such a cycle must exists.

  If $|V(C) \cap A| = k$, then we are done, thus we may assume there
  exists a vertex $a \in A$ with $a \notin C$. Construct a new graph
  $G'$ by duplicating the vertex $a$ $k$ times. Let $A'$ be the $k$
  copies in $G'$ and consider a minimal $A$-$C$ separating set $X$ in
  $G'$. By minimality of $X$, if $A' \cap X \neq \emptyset$, then
  $X = A$ because of the symmetry of elements of $A'$. Otherwise,
  $X \cap A' = \emptyset$ implying that $X$ is a separating set in
  $G$. If $|X| \geq k$, then by Menger's Theorem, there exists $k$
  vertex disjoint paths from $A$ to $C$. By the pigeon hole principle,
  there must exist a path $P$ from $a_i$ to $a_j$ containing no
  vertices in $A$ in $C$ such that two of the disjoint $k$ paths end
  on vertices on $P$. These two paths along with $P$ can be used to
  construct a cycle $C'$ containing $V(C) \cap A$ and $a$,
  contradicting the maximality of $C$. Thus $|X| < k$. This can only
  be the case when $X = V(C)$ since $G$ is $k$-connected. But then by
  Menger's Theorem, there is are $|V(C)|$ edge disjoint paths from
  $A'$ to $C$. Once again these paths can be used to find a cycle $C'$
  containing $V(C) \cap A$ and $a$, contradicting the maximality of
  $C$.
\end{proof}

\begin{problem} Show that for every $k$, there is $N$ such that for
  every $n \geq N$ and every coloring $c$ of ${[n]\choose 2}$ there is
  either a triangle with all edges having distinct colors or a
  sequence $i_1, i_2, \ldots, i_k$ of distinct elements from $[n]$
  such that for every $p \in [k - 1]$,
  $\{i_p, i_{p+1}\}, \cdots \{i_p, i_k\}$ have the same color.
\end{problem}
\begin{proof} Let $R(k)$ denote the minimum value of $N$ for which
  this theorem applies for $k$. Note that for $k = 1$, the claim holds
  for $N = 1$. The rest of the proof will be by induction. Assume
  $k > 1$. Let $G$ be the complete graph on $[n]$ for
  $n > 2R(k-1) + 2{R(k-1)}^2$ and $f$ any coloring of $G$. Let $v$ be
  a vertex in $G$ and $c$ the color that appears the maximum number of
  time among edges incident to $v$. Let $N_c(v)$ denote the
  neighborhood of $v$ using edges with color $c$. If
  $|N_c(v)| \geq R(k-1)$, then by induction $G[N_c(v)]$ contains a
  sequence $i_1, i_2, \cdots i_{k-1}$ of distinct elements satisfying
  the claim, or $G[N_c(v)]$ contains a triangle with all edge colors
  distinct. This implies that either the sequence
  $v, i_1, \cdots i_{k -1}$ exists or a triangle with all edge colors
  distinct exits in $G$, proving the claim true.

  Thus we may assume that no such vertex exists. Then for all vertices
  $v$ and any color $c$, $N_c(v) < R(k-1)$. Let $v$ be a vertex in $G$
  and $A$ a minimum set of size at least $R(k - 1)$ such that for any
  vertex $a \in A$, if the color of $va$ is $c$, then the color of
  $va'$ being $c$ implies that $a' \in A$. By minimality of $A$ and
  since $N_c(v) \leq R(k-1)$, we know that $|A| \leq 2R(k-1) - 1$. Let
  $S = V(G) - \{v\} - A$. Define
  $S' = \{s \in S | \forall a \in A, f(va) \neq f(as)\}$. Note that
  $|S'| \geq |S| - 2{R(k-1)}^2$ since each vertex in $a$ could
  eliminate at most $R(k-1)$ elements from consideration in $S$. Let
  $s \in S'$. If $G$ does not contain a triangle with each edge having
  a distinct color, then for all $a \in A$, $f(as) = f(vs)$ by the
  definition of $S'$. Since $|A| \geq R(k - 1)$ and $N_c(v) < R(k-1)$,
  there must exist a vertex $a \in A$ such that $f(as) \neq
  f(vs)$. But then $vas$ forms a triangle with all edges distinct
  colors. Therefore no such sets $A$ and $S'$ exist. But then we get
  that
  \[
    |G| =|A| + |S| + 1 \leq |A| + |S'| + 2{R(k-1)}^2 + 1 \leq 2R(k-1)
    + 2{R(k-1)}^2 < |G|
  \]
  Thus such a sets $A$ and $S'$ must exist and $G$ contains a triangle
  with all edges having distinct color.
\end{proof}

\end{document}
